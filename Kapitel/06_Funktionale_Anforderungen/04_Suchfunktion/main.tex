\section{Suchfunktion}
\paragraph{Ziel}
Der Benutzer kann nach Gebäuden und Räumen suchen. 
Das Suchen ist möglich mit:
\begin{itemize}
    \item Gebäudenummer
    \item Gebäudenummer und Raumnummer
    \item Name einer Person (wenn nach Büro gesucht wird)
    \item lokaler oder globaler Alias
\end{itemize}

\paragraph{Kategorie}
primär
\paragraph{Verfügbarkeit}
offline (eingeschränkt: Suche mit Personenname ggf. nur online möglich), alle Benutzer
\paragraph{Vorbedingung}
Der Benutzer befindet sich in der Hauptansicht (Kartenansicht).
\paragraph{Nachbedingung (Erfolg)}
Das gesuchte Objekt wird angezeigt.
\paragraph{Nachbedingung (Misserfolg)}
Es erscheint eine Fehlermeldung. Existiert das Gebäude aber der gesuchte Raum nicht, wird zusätzlich das Gebäude angezeigt.
\paragraph{Akteure}
Benutzer
\paragraph{Auslösendes Ereignis}
Der Benutzer klickt auf das Suchfeld.
\paragraph{Beschreibung}
\begin{enumerate}
    \item Die Tastatur öffnet sich. Die App schlägt die letzten Suchbegriffe vor.
    \item Der Benutzer tippt einen Suchbegriff ein.
    \item Die App zeigt Vorschläge zum eingegebenen Text an. Die Vorschläge werden nach jedem eingegebenen Zeichen aktualisiert.
    \item Der Nutzer wählt einen Vorschlag aus oder bestätigt die Eingabe.
    \item Die App prüft, ob zur Eingabe ein Treffer existiert.
    \subitem{Erfolg:} Die App zeigt das gesuchte Gebäude bez. den gesuchten Raum an.
    \subitem{Misserfolg:} 
    \subsubitem Wenn kein Gebäude zum eingegebenen Suchbegriff existiert, oder die eingegebene Person nicht im System existiert, wird eine Fehlermeldung angezeigt.
    \subsubitem Existiert das Gebäude, aber der Raum nicht, zeigt die App das Gebäude und eine Fehlermeldung an.
\end{enumerate}
