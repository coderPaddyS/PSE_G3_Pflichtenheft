\section{Suchfunktion}
\paragraph{Ziel}
Suchen und Finden von Gebäuden, Räumen und Büros durch Namen, Nummer oder Alias
\paragraph{Kategorie}
primär
\paragraph{Vorbedingung}
Der Benutzer befindet sich in der Hauptansicht (Kartenansicht).
\paragraph{Nachbedingung (Erfolg)}
Das gesuchte Objekt wird angezeigt.
\paragraph{Nachbedingung (Misserfolg)}
Es erscheint eine Fehlermeldung. Existiert der gesuchte Raum nicht, aber das Gebäude, wird das Gebäude zusammen mit einer Fehlermeldung angezeigt.
\paragraph{Akteure}

\paragraph{Auslösendes Ereignis}
Benutzer klickt auf das Suchfeld
\paragraph{Beschreibung}
\begin{enumerate}
    \item Der Benutzer klickt auf das Suchfeld.
    \item Die Tastatur öffnet sich. Die App schlägt die letzten Suchbegriffe vor.
    \item Der Benutzer tippt eine Gebäudenummer, eine Gebäudenummer zusammen mit einer Raumnummer, oder den Namen einer Person, die ein Büro besitzt, oder einen Alias ein.
    \item Die App zeigt Vorschläge zum eingegebenen Text an. Die Vorschläge werden nach jedem eingegebenen Zeichen aktualisiert.
    \item Der Nutzer wählt einen Vorschlag durch Klicken darauf aus oder bestätigt die Eingabe mit der Entertaste.
    \item Die App prüft, ob zur Eingabe ein Treffer existiert.
    \subitem{Erfolg:} Die App zeigt das gesuchte Gebäude bez. den gesuchten Raum an.
    \subitem{Misserfolg:} 
    \subsubitem Wenn kein Gebäude zur eingegeben Nummer/Alias existiert, oder die eingegebene Person nicht im System existiert, wird eine Fehlermeldung angezeigt.
    \subsubitem Wenn das eingegebene Gebäude existiert, aber der dazu eingegebene Raum nicht, wird das Gebäude angezeigt. Eine Fehlermeldung wird angezeigt, dass der Raum (noch) nicht im System existiert.
    \item Der eingegebene Text bleibt im Suchfeld stehen.
\end{enumerate}