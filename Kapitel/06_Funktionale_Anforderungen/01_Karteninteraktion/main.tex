\section{Karteninteraktion}
\paragraph{Ziel}
Benutzer können mit der Karte des Campus interagieren.
\paragraph{Kategorie}
primär
\paragraph{Verfügbarkeit}
offline, alle Benutzer
\paragraph{Vorbedingung}

\paragraph{Nachbedingung}

\paragraph{Akteure}
angemeldeter oder nicht angemeldeter Benutzer
\paragraph{Auslösendes Ereignis}
Der Benutzer startet die Applikation oder kehrt aus einem andern Menü zur Kartenansicht zurück.
\paragraph{Anzeigen der Karte}
\begin{enumerate}[start=10, label=\textbf{/FA\arabic*/}, align=left]
    \item Die App zeigt einen Ausschnitt einer Karte. Die verfügbare Karte deckt mindestens einen Radius von 5 km um den Campus Süd ab. Auf der Karte werden Straßen, und Gebäude des Campus angezeigt. Gebäude des Campus sind auf der Karte hervorgehoben. An jedem Gebäude steht die Gebäudenummer.
\end{enumerate}
\paragraph{Interaktionen}
\begin{enumerate}[start=11, label=\textbf{/FA\arabic*/}, align=left]
    \item Der Benutzer kann mit einem Finger die Karte verschieben. Die Karte wird synchron mit dem Finger verschoben. Dabei bleibt die Stelle der Karte, auf der sich der Finger befindet, beim Verschieben unter dem Finger.
    \item Der Benutzer kann durch das Zusammenführen von zwei Fingern aus der Karte herauszoomen. Der Benutzer kann durch das Auseinanderführen von zwei Fingern in die Karte hineinzoomen.
    \item Der Benutzer kann auf Gebäude klicken. Die App zeigt Informationen zum Gebäude an. Die Informationen beinhalten mindestens Gebäudenummer und Adresse. Die Karte ist dabei weiterhin sichtbar. Klickt der Benutzer auf eine andere Stelle auf der Karte verschwinden die Informationen. Wenn Etagenkarten für das Gebäude vorliegen, wechselt die Karte beim (ersten) Klicken auf das Gebäude zusätzlich in den Etagenmodus.
\end{enumerate}