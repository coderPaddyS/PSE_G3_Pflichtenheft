\section{Aliasverwaltungsfunktion}
\label{Aliasverwaltungsfunktion}
\paragraph{Ziel}
Administratoren können Aliasvorschläge verwalten.
\paragraph{Kategorie}
sekundär
\paragraph{Verfügbarkeit}
online
\paragraph{Vorbedingung}
Adminpanel ist geöffnet.
\paragraph{Nachbedingung}
Adminpanel ist geöffnet.
\paragraph{Akteure}
Admin
\paragraph{Auslösendes Ereignis}

\paragraph{Beschreibung}
\begin{enumerate}
    \item Es werden eingereichte Aliasvorschläge mit ihrer Bewertung angezeigt.
    \item Der Administrator kann auswählen, wie viele positive Bewertungen ein Vorschlag haben muss, damit er angezeigt wird.
    \item Der Administrator kann den Standardwert wählen, wie viele positive Bewertungen ein Vorschlag haben muss, damit er angezeigt wird. Der neue Standardwert wird gespeichert.
    \item Der Administrator kann einen Vorschlag annehmen.
    \item Der Administrator kann einen Vorschlag ablehnen.
    \item Der Administrator kann einen Vorschlag ablehnen und auf eine Blacklist setzen.
    \item Der Administrator kann einen beliebigen Begriff auf die Blacklist setzen.
    \item Der Administrator kann die Blacklist einsehen. Begriffe auf der Blacklist werden nicht mehr als Aliasvorschläge akzeptiert.
    \item Der Administrator kann einen Begriff von der Blacklist löschen.
\end{enumerate}