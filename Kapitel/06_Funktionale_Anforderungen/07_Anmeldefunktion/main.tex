\section{Anmeldefunktion}
\label{Anmeldefunktion}
\paragraph{Ziel}
Nicht angemeldete Benutzer können sich anmelden.
\paragraph{Kategorie}
sekundär
\paragraph{Verfügbarkeit}
online, nicht angemeldete Benutzer
\paragraph{Vorbedingung}
Der Benutzer besitzt ein Konto bei einer unterstützten Anmeldeplattform (vorzugsweise Shibboleth Identity Provider).
\paragraph{Nachbedingung}
Der Benutzer besitzt eine eindeutige ID. Der Benutzer ist ein "angemeldeter Benutzer".
\paragraph{Akteure}
nicht angemeldeter Benutzer, Anmeldeplattform
\paragraph{Auslösendes Ereignis}
\begin{itemize}
      \item Der Benutzer hat auf eine "Anmelden"-Schaltfläche geklickt.
      \item Der Benutzer hat versucht eine Aktion auszuführen, die nur angemeldeten Benutzern zur Verfügung steht.
\end{itemize}
\paragraph{Beschreibung}
\begin{enumerate}
      \item Der Benutzer kann den Anmeldevorgang in jedem Schritt abbrechen. Dann kehrt die App zur vorherigen Ansicht zurück.
      \item Die App weist auf den Datenschutz hin.
      \item Der Benutzer akzeptiert den Datenschutz.
      \item Der Benutzer wählt eins der unterstützten Anmeldeverfahren.
      \item Der Benutzer meldet sich über das gewählte Verfahren an.
      \subitem{Erfolg:} Die App speichert eine eindeutige ID für den Benutzer. Die App weist dem Benutzer die Rolle "angemeldeter Benutzer" zu.
      \item Die App zeigt Informationen zum Erfolg oder Misserfolg der Anmeldung an.
      \item Die App kehrt zur vorherigen Ansicht zurück. 
\end{enumerate}