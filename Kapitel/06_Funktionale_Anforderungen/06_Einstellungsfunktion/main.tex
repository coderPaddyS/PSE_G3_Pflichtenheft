\section{Einstellungsfunktion}
\paragraph{Ziel}
Festlegen der Einstellungen der App.
\paragraph{Kategorie}
primär
\paragraph{Vorbedingung}
Der Benutzer befindet sich in der Hauptansicht (Kartenansicht).
\paragraph{Nachbedingung}
Einstellungen sind lokal auf dem Endgerät gespeichert.
\paragraph{Akteure}

\paragraph{Auslösendes Ereignis}
Benutzer klickt auf das Einstellungs-Icon.
\paragraph{Beschreibung}
\begin{enumerate}
    \item Benutzer klickt auf das Einstellungs-Icon.
    \item Eine Ansicht öffnet sich mit den verschiedenen Optionen der Einstellungen, welche der Nutzer durch anklicken auswählen kann.
    \subitem{Sprache} Auf der Ansicht werden die verschiedenen Sprachen angezeigt. (Deutsch und optional Englsich)
    \subitem{Theme} Auf der Ansicht werden die verschiedenen Themes angezeigt.
    \subitem{Suchverlauf} Auf der Ansicht wird die Option Suchverlauf löschen und die Wahl zwischen Suchverlauf speichern oder nicht speichern angezeigt.
    \subitem{Impressum} Auf der Ansicht werden die Informationen zur App angezeigt.
    \subitem{Datenschutz} Auf der Ansicht werden die Informationen zum Datenschutz angezeigt.
    \subitem{Hilfe} Auf der Ansicht werden Erklärungen zu den verschiedenen Funktionen der App angezeigt.
    \item Durch Klicken auf das Zurück-Icon werden wieder die verschiedenen Optionen der Einstellungen angezeigt.
    \item Durch Klicken auf das Exit-Icon wird das Popup-Fenster geschlossen.
\end{enumerate}