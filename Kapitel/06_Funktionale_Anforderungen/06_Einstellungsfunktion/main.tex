\section{Einstellungsfunktion}
\paragraph{Ziel}
Benutzer können die Einstellungen der App festlegen.
\paragraph{Kategorie}
sekundär
\paragraph{Verfügbarkeit}
offline, alle Benutzer
\paragraph{Vorbedingung}
Der Benutzer befindet sich in der Hauptansicht (Kartenansicht).
\paragraph{Nachbedingung}
Einstellungen wurden angepasst.
\paragraph{Akteure}
Benutzer
\paragraph{Auslösendes Ereignis}
Der Benutzer öffnet die Einstellungen.
\paragraph{Beschreibung}
\begin{enumerate}
    \item Der Benutzer kann eine der unterstützten Sprachen auswählen. Alle Texte in der App werden in der ausgewählten Sprache angezeigt.
    \item Der Benutzer kann ein Anzeige-Theme auswählen.
    \item Der Benutzer kann den Suchverlauf löschen.
    \item Der Benutzer kann die Speicherung des Suchverlaufes aktivieren und deaktivieren. Der Standartwert ist "aktiviert".
    \item Der Benutzer kann sich das Impressum der App anzeigen lassen.
    \item Der Benutzer kann sich Informationen zum Datenschutz der App anzeigen lassen.
    \item Der Benutzer kann sich Erklärungen zu verschiedenen Funktionen der App anzeigen lassen.
\end{enumerate}