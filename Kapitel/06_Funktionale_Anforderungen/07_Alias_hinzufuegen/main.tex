\section{/FA70/ Alias hinzufügen}
\label{/FA70/}
\paragraph{Ziel}
Der Benutzer kann einen Alias lokal hinzufügen (nur für den Benutzer selbst sichtbar). Der Benutzer kann einen Alias vorschlagen (für alle Benutzer sichtbar).
\paragraph{Kategorie}
sekundär
\paragraph{Verfügbarkeit}
\begin{itemize}
    \item Lokales hinzufügen: offline, alle Benutzer
    \item Vorschlagen: online, nur angemeldete Benutzer
\end{itemize}

\paragraph{Vorbedingung}
Gebäude oder Raum ist ausgewählt.
\paragraph{Nachbedingung}
Alias ist lokal gespeichert und wenn gewünscht zusätzlich als Vorschlag eingereicht.
\paragraph{Akteure}
angemeldeter oder nicht angemeldeter Benutzer
\paragraph{Auslösendes Ereignis}
Benutzer wählt das Hinzufügen eines Alias aus.
\paragraph{Beschreibung}
\begin{enumerate}
    \item Die Tastatur öffnet sich.
    \item Der Benutzer gibt den gewünschten Alias ein.
    \item Der Benutzer wählt lokales Hinzufügen oder zusätzliches Vorschlagen als Alias für alle Benutzer aus.
    \item Die App prüft, ob der Alias bereits existiert.
    \subitem{Alias existiert:} Die App zeigt eine Fehlermeldung an.
    \subitem{Alias existiert nicht:} Alias wird lokal gespeichert und, wenn ausgewählt, als Vorschlag an den Server gesendet.
\end{enumerate}
