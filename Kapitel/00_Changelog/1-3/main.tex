\subsection*{Änderungen V1.3 im Vergleich zu V1.2}
\paragraph*{Änderungen in der Qualitätssicherung:}
\begin{itemize}
    \item Verbesserung der Kartenbedienbarkeit in \hyperref[/FA24/]{/FA24/}. Nun schließt sich die Etagenkarte nicht mehr, wenn der Benutzer neben einen Raum klickt. Der Testfall \hyperref[/T121/]{/T121/} wurde entsprechend angepasst.
    \item Redundante Funktionssequenz /T190/ entfernt.
    \item Testfall \hyperref[/T320/]{/T320/} klarer formuliert.
    \item Testfall \hyperref[/T370/]{/T370/} klarer formuliert.
\end{itemize}

\paragraph*{Änderungen durch das Implementierungsdokument:}
\begin{itemize}
    \item Verkleinerung des Kartenausschnittes von einem Radius von 5km um den \Gls{Campus} zur notwendigen Abdeckung aller Gebäude des \Gls{Campus} Süd in \hyperref[/FA11/]{/FA11/}
    \item Entfernung der funktionalen Anforderung /FA15/: \\ Der \Gls{Benutzer} kann die \Gls{Karte} mit zwei Fingern drehen. Es wird ein Kompass angezeigt, der die Himmelsrichtung relativ zu \Gls{Karte} anzeigt.
    \item Entfernung des Testfalls /T181/ zu /FA15/.
    \item Die Anzeigesprache wird nicht mehr in der App eingestellt, sondern ist abhängig von der Systemsprache \hyperref[/FA51/]{/FA51/}.
    \item Entfernung des Testteils Sprache von dem Test \hyperref[/T50/]{/T50/}.
    \item Hinzufügen eines Tests für die Sprache \hyperref[/T51/]{/T51/}.
\end{itemize}