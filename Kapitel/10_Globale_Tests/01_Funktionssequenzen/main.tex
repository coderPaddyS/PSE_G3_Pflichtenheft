\section{Funktionssequenzen}

Für alle Funktionssequenzen gilt:
Offlinefunktionalität ist auch bei Online-Nutzung vorhanden und Funktionalität, für die man nicht angemeldet sein muss, ist auch bei Anmeldung vorhanden. Dabei ist der Anmeldeprozess die einzige Ausnahme.
Folgende Funktionssequenzen sind zu überprüfen:

\paragraph{Benutzer offline}
\begin{enumerate}[label=\textbf{/T\arabic*0/}, align=left]
	\item Suchen von Gebäuden, Räumen durch Standardbezeichner.
	\item Suchen von Gebäuden, Räumen durch lokal hinzugefügte Aliasse.
	\item Suchen von Gebäuden, Räumen durch bereits heruntergeladene, für alle Benutzer sichtbare Aliasse.
	\item Aliasse für Gebäude, Räume lokal auf dem Smartphone hinzufügen.
	\item Auswahl von Kategorien Themes, Sprachen, Suchverlauf, Impressum, Hilfe.
	\item Löschen des Suchverlaufs.
	\item Entfernen von lokal gespeicherten Aliassen.
	\item Anfordern des Standortzugriffs.
	\item Anzeigen der Kartenansicht.
	\item Anzeigen des aktuellen Standorts.
	\item Anzeigen von Straßennamen und Kreuzungen beim Anzeigen des aktuellen Standorts.
	\item Anzeige und Interaktion der Etagenkarten.
	\item Auswählen von Gebäuden auf der Karte.
	\item Auswählen von Räumen auf der Etagenkarten.
	\item Wechseln der Etagen auf der Etagenkarte.
	\item Verschieben der Karte mit einem Finger.
	\item Hineinzoomen in die Karte durch Auseinanderführen von zwei Fingern.
	\item Herauszoomen aus der Karte durch Zusammenführen von zwei Fingern.
	\item Anzeigen von lokal gespeicherten Informationen zu dem ausgewählten Gebäude bzw. Raum.
\end{enumerate}

\paragraph{Benutzer online, nicht angemeldet}
\begin{enumerate}[label=\textbf{/T\arabic*0/}, align=left]
	\item Suchen von Gebäuden, Räumen durch für alle Benutzer sichtbare Aliasse.
	\item Suchen von Räumen über Personennamen.
	\item Anzeigen von allen Informationen zu dem ausgewählten Gebäude bzw. Raum.
	\item Anmelden.
\end{enumerate}

\paragraph{Benutzer online, angemeldet}
\begin{enumerate}[label=\textbf{/T\arabic*0/}, align=left]
	\item Vorschläge für Aliasse von Gebäuden, Räumen hinzufügen.
	\item Vorschläge für Aliasse von Gebäuden, Räumen bewerten.
\end{enumerate}

\paragraph{Admin online}
\begin{enumerate}[label=\textbf{/T\arabic*0/}, align=left]
	\item Standardwert der Mindestanzahl positiver Bewertungen der Aliasvorschläge festlegen.
	\item Temporär die Mindestanzahl positiver Bewertungen der Aliasvorschläge auswählen.
	\item Aliasvorschläge einsehen, annehmen, ablehnen, auf die Blacklist setzen.
	\item Blacklist einsehen, Begriffe hinzufügen und entfernen.
\end{enumerate}
