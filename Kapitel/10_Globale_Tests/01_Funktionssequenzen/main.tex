\section{Funktionssequenzen}

Für alle Funktionssequenzen gilt:
Offlinefunktionalität ist auch bei Online-Nutzung vorhanden und Funktionalität, für die man nicht angemeldet sein muss, ist auch bei Anmeldung vorhanden. Dabei ist der Anmeldeprozess die einzige Ausnahme.
Folgende Funktionssequenzen sind zu überprüfen:

\paragraph{%TODO:\gls{Benutzer}END 
	offline}
\begin{enumerate}[label=\textbf{/T\arabic*0/}, align=left]
	\item Suchen von Gebäuden, Räumen durch %TODO:\glspl{Standardbezeichner}END 
		.
	\item Suchen von Gebäuden, Räumen durch lokal hinzugefügte %TODO:\glspl{Alias}END 
		.
	\item Suchen von Gebäuden, Räumen durch bereits heruntergeladene, für alle Benutzer sichtbare %TODO:\glspl{Alias}END 
		.
	\item Aliasse für Gebäude, Räume lokal auf dem Smartphone hinzufügen.
	\item Auswahl von Kategorien Themes, Sprachen, Suchverlauf, Impressum, Hilfe.
	\item Löschen des Suchverlaufs.
	\item Entfernen von lokal gespeicherten %TODO:\glspl{Alias}END 
		n.
	\item Anfordern des Standortzugriffs.
	\item Anzeigen der %TODO:\gls{Kartenansicht}END 
		.
	\item Anzeigen des aktuellen Standorts.
	\item Anzeigen von Straßennamen und Kreuzungen beim Anzeigen des aktuellen Standorts.
	\item Anzeige und Interaktion der %TODO:\gls{Etagenkarte}END 
		.
	\item Auswählen von Gebäuden auf der %TODO:\gls{Karte}END 
		.
	\item Auswählen von Räumen auf der %TODO:\gls{Etagenkarte}END 
		.
	\item Wechseln der Etagen auf der %TODO:\gls{Etagenkarte}END 
		.
	\item Verschieben der %TODO:\gls{Karte}END 
		mit einem Finger.
	\item Hineinzoomen in die %TODO:\gls{Karte}END 
		durch Auseinanderführen von zwei Fingern.
	\item Herauszoomen aus der %TODO:\gls{Karte}END 
		durch Zusammenführen von zwei Fingern.
	\item Anzeigen von lokal gespeicherten Informationen zu dem ausgewählten Gebäude bzw. Raum.
\end{enumerate}

\paragraph{%TODO:\gls{Benutzer}END 
	online, nicht angemeldet}
\begin{enumerate}[label=\textbf{/T\arabic*0/}, align=left]
	\item Suchen von Gebäuden, Räumen durch für alle %TODO:\glspl{Benutzer}END 
		sichtbare %TODO:\glspl{Alias}END 
		.
	\item Suchen von Räumen über Personennamen.
	\item Anzeigen von allen Informationen zu dem ausgewählten Gebäude bzw. Raum.
	\item Anmelden.
\end{enumerate}

\paragraph{%TODO:\gls{Benutzer}END 
	online, angemeldet}
\begin{enumerate}[label=\textbf{/T\arabic*0/}, align=left]
	\item Vorschläge für %TODO:\glspl{Alias}END 
		von Gebäuden, Räumen hinzufügen.
	\item Vorschläge für %TODO:\glspl{Alias}END 
		von Gebäuden, Räumen bewerten.
\end{enumerate}

\paragraph{%TODO:\gls{Administrator}END 
	online}
\begin{enumerate}[label=\textbf{/T\arabic*0/}, align=left]
	\item Standardwert der Mindestanzahl positiver Bewertungen der %TODO:\gls{Alias}END 
		vorschläge festlegen.
	\item Temporär die Mindestanzahl positiver Bewertungen der %TODO:\gls{Alias}END 
		vorschläge auswählen.
	\item %TODO:\gls{Alias}END 
		vorschläge einsehen, annehmen, ablehnen, auf die %TODO:\gls{Blacklist}END 
		setzen.
	\item %TODO:\gls{Blacklist}END 
		einsehen, Begriffe hinzufügen und entfernen.
\end{enumerate}
