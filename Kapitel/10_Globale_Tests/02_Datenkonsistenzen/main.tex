\section{Datenkonsistenzen}

Folgende Datenkonsistenzen sind einzuhalten:
\begin{enumerate}[label=\textbf{/T\arabic*0/}, align=left, resume]
	\item Es werden bei der Suchfunktion nur Alias von Gebäuden und Räumen anerkannt und somit gefunden, die bereits vom Server angenommen bzw. lokal hinzugefügt wurden.
	\item Existiert der vom Nutzer eingegebene Bezeichner für ein Gebäude nicht, erscheint eine Fehlermeldung.
	\item Existiert der vom Nutzer eingegebene Bezeichner für einen Raum nicht, erscheint eine Fehlermeldung.
	\item Die Bezeichner bzw. Aliasse eines Gebäudes bzw. eines Raumes sind insofern eindeutig, dass kein anderer Gebäude oder Raum mit diesen Bezeichnern bzw. Aliassen identifiziert und somit gefunden wird.
	\item Nutzer können nur dann existierende Vorschläge für Aliasse bewerten, wenn sie eingeloggt sind.
	\item Nutzer können nur dann Vorschläge für Aliasse hinzufügen, wenn sie eingeloggt sind.
	\item Ein Vorschlag für einen Alias sollte nicht mehrfach im System existieren.
	\item Ein Nutzer sollte einen Vorschlag nur einmal bewerten können.
	\item Es werden nur Vorschläge für die Eingabe des Nutzers bei der Suchfunktion unterbreitet, die das eingegebene Wort vervollständigen oder Rechtschreibfehler korrigieren und die vorgeschlagene Zeichenkette tatsächlich auch einen Bezeichner für ein Gebäude oder einen Raum ist.
\end{enumerate}