\section{Datenkonsistenzen}

Folgende Datenkonsistenzen sind einzuhalten:
\begin{enumerate}[label=\textbf{/T\arabic*0/}, align=left, resume]
	\item Es werden bei der Suchfunktion nur %TODO:\gls{Alias}END 
		von Gebäuden und Räumen anerkannt und somit gefunden, die bereits vom %TODO:\gls{Server}END 
		angenommen bzw. lokal hinzugefügt wurden.
	\item Existiert der vom %TODO:\gls{Benutzer}END 
		 eingegebene %TODO:\gls{Bezeichner}END 
		 für ein Gebäude nicht, erscheint eine Fehlermeldung.
	\item Existiert der vom %TODO:\gls{Benutzer}END 
		 eingegebene %TODO:\gls{Bezeichner}END 
		 für einen Raum nicht, erscheint eine Fehlermeldung.
	\item Die %TODO:\gls{Bezeichner}END 
		bzw. %TODO:\glspl{Alias}END 
		eines Gebäudes bzw. eines Raumes sind insofern eindeutig, dass kein anderer Gebäude oder Raum mit diesen Bezeichnern bzw. %TODO:\glspl{Alias}END 
		n identifiziert und somit gefunden wird.
	\item %TODO:\glspl{Benutzer}END 
		 können nur dann existierende %TODO:\glspl{Vorschlag}END 
		 für %TODO:\glspl{Alias}END 
		bewerten, wenn sie eingeloggt sind.
	\item %TODO:\glspl{Benutzer}END 
		 können nur dann %TODO:\glspl{Vorschlag}END 
		 für %TODO:\glspl{Alias}END 
		hinzufügen, wenn sie eingeloggt sind.
	\item Ein %TODO:\gls{Vorschlag}END 
		für einen %TODO:\gls{Alias}END 
		sollte nicht mehrfach im System existieren.
	\item Ein %TODO:\gls{Benutzer}END
		sollte einen %TODO:\gls{Vorschlag}END 
		nur einmal bewerten können.
	\item Es werden nur %TODO:\glspl{Vorschlag}END 
		für die Eingabe des %TODO:\gls{Benutzer}END 
		s bei der Suchfunktion unterbreitet, die das eingegebene Wort vervollständigen oder Rechtschreibfehler korrigieren und die vorgeschlagene Zeichenkette tatsächlich auch einen %TODO:\gls{Bezeichner}END 
		für ein Gebäude oder einen Raum ist.
\end{enumerate}