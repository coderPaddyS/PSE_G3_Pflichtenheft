\newpage
\section{Admin-Panel}
\label{Admin-Panel}

\paragraph{Bedeutung}
Das \Gls{Admin-Panel} ist die zentrale graphische Benutzeroberfläche für \Glspl{Administrator}.

\paragraph{Ziel}
% TODO: Kriterien verlinken
\Glspl{Administrator} authentifizieren sich.
\Glspl{Administrator} sehen \Glspl{Alias-Vorschlag} ein.
\Glspl{Administrator} akzeptieren final \Glspl{Alias-Vorschlag}.
\Glspl{Administrator} lehnen final \Glspl{Alias-Vorschlag} ab.
\Glspl{Administrator} fügen final \Glspl{Alias-Vorschlag} der \Gls{Blacklist} hinzu.
\Glspl{Administrator} löschen final \Glspl{Alias-Vorschlag}.
\Glspl{Administrator} bearbeiten die \Gls{Blacklist}.
\Glspl{Administrator} können gemäß \ref{/KK10/} Etagenkarten hinzufügen. Dies kann gemäß \ref{/KK20/} unterstützt werden.
\Glspl{Administrator} können gemäß \ref{/KK30/} genutzte \Gls{API}s aktualisieren und ändern.

\paragraph{Personengruppe}
Ausschließlich \Glspl{Administrator} haben Zugriff auf die \Gls{Schnittstelle}.

\paragraph{Plattform}
Das \Gls{Admin-Panel} wird im \Gls{Browser} (\Gls{Firefox} und \Gls{Chromium}) dargestellt.
Hierfür muss das \Gls{Endgeraet} \Gls{HTML5}, \Gls{CSS3} und \Gls{JavaScript} ausführen können.

\paragraph{Erreichbarkeit}
Das \Gls{Admin-Panel} ist nur über das Internet erreichbar. 
Das \Gls{Admin-Panel} ist erreichbar unter \texttt{https://pse.itermori.de/admin}.

\paragraph{Interaktion mit Server}
Das \Gls{Admin-Panel} kommuniziert durch \Gls{HTTPS} mit dem \Gls{Server} über eine selbstdefinierte \Gls{API}.
Diese \Gls{API} ist vorzugsweise eine \Gls{GraphQL}-\Gls{API}, alternativ eine \Gls{REST}ful-\Gls{API}.

\paragraph{Beschreibung}
\begin{enumerate}
    \item Die \Gls{Schnittstelle} ist passwortgeschützt.
    \item \Glspl{Administrator} können aktuelle \Glspl{Alias-Vorschlag} einsehen, annehmen und ablehnen.
\end{enumerate}