\section{App}
\label{App-UI}
\paragraph{Ziel}
\Glspl{Benutzer} können sich auf dem \Gls{Campus} zurechtfinden.
\paragraph{Personengruppe}
Zielgruppe der Applikation
\paragraph{Plattform}
\Gls{Android} Applikation
\paragraph{Erreichbarkeit}
Die App kann sowohl mit als auch ohne Internetverbindung benutzt werden. Im \Gls{Offlinemodus} sind keine Funktionen verfügbar, die eine Verbindung zum \Gls{Server} benötigen. Im \Gls{Offlinemodus} sind alle anderen Funktionen verfügbar. Insbesondere sind Kartennavigation, Etagennavigation, Ortungsfunktion und die Suchfunktion verfügbar.

\paragraph{Beschreibung}
Die Hauptansicht der App ist die \Gls{Kartenansicht}. Alle weiteren Funktionen sind von hier aus erreichbar.

\subsection*{Beispielansicht}
Die Benutzeroberfläche der App könnte zum Beispiel so aussehen:

\begin{figure}[H]
    \centering
    \begin{minipage}[b]{0.4\textwidth}
        \includegraphics[width=5cm]{\relimgfile{Hauptansicht.png}}
        \caption{\Gls{Kartenansicht}}
    \end{minipage}
    \hfill
    \begin{minipage}[b]{0.4\textwidth}
        \includegraphics[width=5cm]{\relimgfile{Gebäudeinformationen_mit_Etagenansicht.png}}
        \caption{\Gls{Etagenkartenansicht}}
    \end{minipage}
\end{figure}

\paragraph{Beschreibung Bild 8.1}
Das Bild zeigt die \Gls{Kartenansicht} der App.
\begin{itemize}
    \item Am oberen Bildschirmrand befindet sich die Suchleiste.
    \item Oben links befindet sich der Button um die Einstellungen zu öffnen.
    \item Unten rechts befindet sich der Button zur \Gls{Ortung}.
\end{itemize}
\paragraph{Beschreibung Bild 8.2}
Das Bild zeigt eine mögliche Ansicht nach Klicken auf das Gebäude 50.34.
\begin{itemize}
    \item Im unteren Drittel der Ansicht werden Informationen zum Gebäude angezeigt.
    \item Auf dem Gebäude wird die \Gls{Etagenkarte} des 1. Obergeschosses angezeigt.
    \item Am rechten Rand unter der Suchleiste befindet sich eine Schaltfläche. Hier wird das aktuell zu sehende Stockwerk (1. OG) angezeigt. Um diese Anzeige herum befinden sich Buttons zum Wechseln der Etage nach oben und nach unten.
\end{itemize}
Kartenquelle: \href{https://www.openstreetmap.org/}{www.openstreetmap.org}
