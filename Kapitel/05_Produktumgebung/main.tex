\newpage
\chapter{Produktumgebung}

\section{Hard- und Software}
    \paragraph{App}
        Die App läuft auf einem Android Smartphone, welches mindestens API-Level 22 unterstützt.
        Das Smartphone besitzt einen Internetzugang.
        Das Smartphone ist dazu in der Lage das \textbf{G}lobal \textbf{P}ositioning \textbf{S}ystem %TODO:\Gls{GPS}END
            zu nutzen.

    \paragraph{Backend}
        Das Server-Backend läuft auf einer virtuellen Maschine auf einem Server.
        Auf der virtuellen Maschine läuft Debian 10 (Buster).
        Das Server-Backend ist vom darunterliegenden System durch eine eigene virtuelle Linux-Umgebung, hier Docker, abgegrenzt.
        Die virtuelle Linux-Umgebung ist fähig, Java-Programme durch eine JVM auszuführen.
        Das Backend ist in Java 17 geschrieben.

    \paragraph{Admin-Panel}
        Das Admin-Panel ist auf jedem HTML5-, JavaScript- und CSS-fähigem Gerät aufrufbar.

\section{Schnittstellen}
    \paragraph{Netzwerk-Zugriffe}
        Die App sowie das Admin-Panel kommunizieren mit dem Server über HTTPS. Dieser ist über die definierte URL \textit{https://pse.itermori.de},
        das Admin-Panel über die URL \textit{https://pse.itermori.de/admin}, erreichbar. \\
        Der Server kommuniziert durch HTTPS mit den KIT-Servern erreichbar unter \textit{https://kit.edu}.
