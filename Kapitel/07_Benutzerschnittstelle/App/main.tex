\section{App}
\label{App-UI}
\paragraph{Ziel}
Benutzer können sich auf dem Campus zurechtfinden.
\paragraph{Personengruppe}
Zielgruppe der Applikation
\paragraph{Plattform}
Android Applikation
\paragraph{Erreichbarkeit}
Die App kann sowohl mit als auch ohne Internetverbindung benutzt werden. Im Offlinemodus sind keine Funktionen verfügbar, die eine Verbindung zum Server benötigen. Im Offlinemodus sind alle anderen Funktionen verfügbar. Insbesondere sind Kartennavigation, Etagennavigation, Ortungsfunktion und die Suchfunktion verfügbar.

\paragraph{Beschreibung}
Die Hauptansicht der App ist die Kartenansicht. Alle weiteren Funktionen sind von hier aus bedienbar.

\begin{figure}[H]
    \centering
    \begin{minipage}[b]{0.4\textwidth}
        \includegraphics[width=5cm]{\relimgfile{Hauptansicht.png}}
        \caption{Kartenansicht}
    \end{minipage}
    \hfill
    \begin{minipage}[b]{0.4\textwidth}
        \includegraphics[width=5cm]{\relimgfile{Gebäudeinformationen_mit_Etagenansicht.png}}
        \caption{Etagenansicht}
    \end{minipage}
\end{figure}

