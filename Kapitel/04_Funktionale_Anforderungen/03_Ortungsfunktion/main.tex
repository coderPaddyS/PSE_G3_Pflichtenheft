\section{Ortungsfunktion}
\paragraph{Ziel}
Einfacheres Zurechtfinden durch Anzeigen des aktuellen Standorts auf der Karte
\paragraph{Kategorie}
primär
\paragraph{Vorbedingung}
Der Benutzer befindet sich in der Hauptansicht (Kartenansicht).
\paragraph{Nachbedingung}
Der Benutzer befindet sich in der Hauptansicht (Kartenansicht).
\paragraph{Akteure}

\paragraph{Auslösendes Ereignis 1}
Der Benutzer klickt auf den Knopf zur Ortung.
\paragraph{Beschreibung 1}
\begin{enumerate}
    \item Wenn die App keinen Standortzugriff besitzt, wird dieser angefordert.
    \item Wird der Standortzugriff verweigert, passiert nichts und die normale Kartenansicht bleibt ohne Änderung bestehen.
    \item Erhält die App Standortzugriff oder besitzt sie ihn bereits, wird der aktuelle Standort auf ihr angezeigt. Die Karte wird so verschoben, dass sich der aktuelle Standort in der Mitte befindet. Es wird soweit in die Karte hineingezoomt, dass Straßennamen und Kreuzungen erkennbar sind.
\end{enumerate}

\paragraph{Auslösendes Ereignis 2}
Der Benutzer öffnet die App und gelangt das erste Mal (für diese Session) in die Kartenansicht.
\paragraph{Beschreibung 2}
\begin{enumerate}
    \item Wenn die App keinen Standortzugriff hat, wird die Position auf der Karte angezeigt, die zuletzt angezeigt wurde, bevor die App geschlossen wurde. Liegt keine letzte Position vor, wird der Platz zwischen der Mensa am Adenauerring und der Bibliothek angezeigt.
    \item Besitzt die App Standortzugriff, wird der aktuelle Standort auf ihr angezeigt. Die Karte wird so angezeigt, dass sich der aktuelle Standort in der Mitte befindet und Straßennamen und Kreuzungen erkennbar sind.
\end{enumerate}