\section{Etagennavigation}
\paragraph{Ziel}
Navigieren und Interagieren mit der Etagenkarte eines Gebäudes
\paragraph{Kategorie}
primär
\paragraph{Vorbedingung}
Für das ausgewählte Gebäude liegen Etagenkarten vor.
\paragraph{Nachbedingung}

\paragraph{Akteure}

\paragraph{Auslösendes Ereignis}
Der Benutzer klickt auf ein Gebäude oder sucht einen Raum über die Suchfunktion.
\paragraph{Beschreibung der Karte}
\begin{enumerate}
    \item Die Etagenkarte eines Gebäudes wird über dem Gebäude eingeblendet. Die Karte enthält, wenn verfügbar die verschiedenen Räume auf der Etage. Gegebenenfalls kann eine Etagenkarte auch nur die Hörsäle eines Gebäudes anzeigen.
    \item Existieren für mehrere Etagen des Gebäudes Etagenkarten, wird eine Schaltfläche zum wechseln der Etage und ein Text mit der aktuellen Etage angezeigt.
\end{enumerate}
\paragraph{Beschreibung der Interaktionen}
\begin{enumerate}
    \item Klickt der Benutzer auf einen Raum, so zeigt die App am unteren Bildschirmrand Informationen zu diesem Raum wie Raumnummer und um wessen Büro es sich handelt (sofern diese Information vorliegt) an. Klickt der Benutzer erneut (ggf. auf eine andere Stelle) auf die Karte schließt sich das Informationsfenster.
    \item Klickt der Benutzer neben das Gebäude auf die Karte schließt sich die Etagenansicht und die normale Kartenansicht wird angezeigt. Kehrt der Benutzer direkt wieder in die Etagenansicht des gleichen Gebäudes zurück, so wird die Etagenkarte der Etage angezeigt, die vor dem Verlassen angezeigt wurde.
    \item Klickt der Benutzer auf den Knopf zum Wechseln der Etage nach oben, so wird die Etagenkarte der darüberliegenden Etage angezeigt. Das gleiche Vorgehen funktioniert entsprechend auch nach unten. Wird gerade die Etagenkarte der obersten (bez. untersten) Etage angezeigt, so ist der Knopf nach oben (bez. unten) visuell erkennbar deaktiviert und ohne Funktion.
\end{enumerate}