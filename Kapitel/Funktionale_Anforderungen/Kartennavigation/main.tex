\section{Kartennavigation}
\label{Kartennavigation}
\paragraph{Ziel}
Navigieren und Interagieren mit der Karte des Campus
\paragraph{Kategorie}
primär
\paragraph{Vorbedingung}

\paragraph{Nachbedingung}

\paragraph{Akteure}

\paragraph{Auslösendes Ereignis}
Der Benutzer startet die Applikation oder kehrt aus einem andern Menü zur Kartenansicht zurück.
\paragraph{Beschreibung der Karte}
\begin{enumerate}
    \item Die Basis für die Karte ist eine Straßenkarte, die mindestens einen Radius von 30 km um den Campus Süd enthält.
    \item Auf der Karte sind die Gebäude des Campuses erkennbar hervorgehoben. Auf/neben jedem Gebäude steht die Gebäudenummer.
\end{enumerate}
\paragraph{Beschreibung der Interaktionen}
\begin{enumerate}
    \item Der Benutzer kann mit einem Finger die Karte verschieben. Die Karte wird synchron mit dem Finger verschoben, d.h. die Stelle auf der Karte, auf der sich der Finger befindet bleibt beim Verschieben unter dem Finger.
    \item Der Benutzer kann durch das Zusammenführen von zwei Fingern aus der Karte herauszoomen. Der Benutzer kann durch das Auseinanderführen von zwei Fingern in die Karte hineinzoomen.
    \item Klickt der Benutzer auf ein Gebäude werden in der unteren Bildschirmhälfte Informationen zum Gebäude angezeigt. Die Karte ist dabei weiterhin sichtbar. Klickt der Benutzer auf eine andere Stelle auf der Karte verschwinden die Informationen. Wenn Etagenkarten für das Gebäude vorliegen wechselt die Karte beim (ersten) Klicken auf das Gebäude zusätzlich in den Etagenmodus.
\end{enumerate}