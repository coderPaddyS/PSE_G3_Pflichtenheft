\subsection{Anmeldefunktion}
\label{Anmeldefunktion}
\paragraph{Ziel}
Identifikation eines Nutzers durch eine eindeutig bestimmte ID ohne genaueren Informationen über den Benutzer zu speichern.
\paragraph{Kategorie}
sekundär
\paragraph{Vorbedingung}
Der Benutzer befindet sich auf einer Gebäudeansicht und möchte einen Alias vorschlagen oder Bewerten
\paragraph{Nachbedingung}
Der Benutzer hat eine eindeutig bestimmte ID und kann nun Aliase vorschlagen und bewerten.
\paragraph{Akteure}

\paragraph{Auslösendes Ereignis}
Der Benutzer hat auf die "Anmelden"-Schaltfläche gedrückt oder wollte einen Alias bewerten oder vorschlagen.
\paragraph{Beschreibung}
\begin{enumerate}
    \item Der Benutzer hat das Ereignis ausgelöst
    \item Der Benutzer bekommt beim erstmaligen Auslösen einen Hinweis über das verwendete Anmeldeverfahren und die betroffenen und gesammelten Daten.
          Der Benutzer wird gefragt ob er fortfahren möchte.
          \subtitem Lehnt der Nutzer ab, so wird er zurück zur Gebäudeansicht geleitet.
    \item Der Benutzer hat akzeptiert und das entsprechende Anmeldeverfahren wird eingeleitet.
          \subitem Schlägt das Anmeldeverfahren fehl oder bricht der Nutzer ab, so wird er zurück zur Gebäudeansicht geleitet.
    \item Nach erfolgreicher Anmeldung wird der Nutzer zur Gebäudeansicht zurückgeleitet und kann nun Aliase bewerten oder vorschlagen.
\end{enumerate}