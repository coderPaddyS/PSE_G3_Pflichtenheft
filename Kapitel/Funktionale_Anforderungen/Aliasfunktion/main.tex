\subsection{Aliasfunktion}
\label{Aliasfunktion}
\paragraph{Ziel}
Hinzufügen eines Alias zu einem Gebäude oder Raum.
\paragraph{Kategorie}
sekundär
\paragraph{Vorbedingung}
Gebäude bzw. Raum existiert und ist ausgewählt.
\paragraph{Nachbedingung}
Alias ist zusätzlicher Bezeichner des Gebäudes bzw. Raums.
\paragraph{Akteure}

\paragraph{Auslösendes Ereignis}
Benutzer klickt auf das Aliashinzufügefeld.
\paragraph{Beschreibung}
\begin{enumerate}
    \item Benutzer klickt auf das Aliashinzufügefeld.
    \item Die Tastatur öffnet sich.
    \item Der Benutzer gibt den gewünschten Alias ein.
    \item Der Benutzer bestätigt die Eingabe mit der Entertaste.
    \item Die App prüft, ob die Eingabe bereits ein Bezeichner für ein Gebäude, einen Raum oder eine Person ist.
    \subitem{Erfolg:} Die App zeigt eine Fehlermeldung unter dem Aliashinzufügefeld an.
    \subitem{Misserfolg:} Die Tasstatur schließt sich.
    \subitem Ein Popup-Fenster öffnet sich mit der Wahl zwischen nur lokalem Hinzufügen oder zusätzlichem Vorschlagen als Bezeichner für das Gebäude bzw. den Raum für alle Benutzer.
    \subitem Der Benutzer wählt eine der Möglichkeiten durch anklicken. 
    \subitem Das Popup-Fenster schließt sich.
    \subitem Bei der Wahl mit Vorschlag wird der Alias mit zugehörigem Gebäude bzw. Raum auf den Server geladen.
    \subitem Der Alias wird als Bezeichner zu dem ausgewählten Gebäude bzw. Raum hinzugefügt und lokal auf dem Endgerät gespeichert.
\end{enumerate}