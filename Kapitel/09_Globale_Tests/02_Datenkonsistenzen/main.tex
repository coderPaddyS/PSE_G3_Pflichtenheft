\section{Datenkonsistenzen}

Folgende Datenkonsistenzen sind einzuhalten:
\begin{enumerate}[label=\textbf{/T\arabic*0/}, align=left, resume]
	\item \label{/T300/} Testfall zu \hyperref[/FA40/]{/FA40/}: Es werden bei der Suchfunktion nur \Glspl{Alias} von Gebäuden und Räumen anerkannt und somit gefunden, die bereits vom \Gls{Server} angenommen bzw. \gls{lokal} hinzugefügt wurden.
	\item \label{/T310/} Testfall zu \hyperref[/FA40/]{/FA40/}: Existiert der vom \Gls{Benutzer} eingegebene Bezeichner für ein Gebäude bzw. einen Raum nicht, erscheint eine Fehlermeldung.
	\item \label{/T320/} Testfall zu \hyperref[/FA70/]{/FA70/}: Bezeichner bzw. \Glspl{Alias} eines Gebäudes bzw. eines Raumes sind insofern eindeutig, dass kein anderes Gebäude oder Raum mit diesen Bezeichnern bzw. \Glspl{Alias}n identifiziert und somit gefunden wird.
	\item \label{/T330/} Testfall zu \hyperref[/FA90/]{/FA90/}: \Glspl{Benutzer} können nur dann \Glspl{Alias-Vorschlag} bewerten, wenn sie angemeldet sind.
	\item \label{/T340/} Testfall zu \hyperref[/FA70/]{/FA70/}: \Glspl{Benutzer} können nur dann \glspl{global} \Glspl{Alias-Vorschlag} hinzufügen, wenn sie angemeldet sind.
	\item \label{/T350/} Testfall zu \hyperref[/FA70/]{/FA70/}: Ein \Gls{Alias-Vorschlag} kann nicht mehrfach im System existieren.
	\item \label{/T360/} Testfall zu \hyperref[/FA90/]{/FA90/}: Ein \Gls{Benutzer} kann einen \Gls{Alias-Vorschlag} nur einmal bewerten.
	\item \label{/T370/} Testfall zu \hyperref[/FA40/]{/FA40/}: Es werden nur \Glspl{Alias-Vorschlag} für die Eingabe des \Gls{Benutzer}s bei der Suchfunktion unterbreitet, die das eingegebene Wort vervollständigen oder Rechtschreibfehler korrigieren und die vorgeschlagene Zeichenkette tatsächlich auch einen Bezeichner für ein Gebäude oder einen Raum ist.
\end{enumerate}