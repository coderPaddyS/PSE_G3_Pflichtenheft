\subsection*{NFL30}

% Hier müsste noch irgendwie nicht aktiviertes GPs eingearbeitet werden
Innerhalb einer Android-Emulation wird der emulierte Standort zufällig auf eine Position auf dem Campus Süd geändert.
Anschließend wird die App sowie eine interne Stoppuhr gestartet.
Der durch die App gespeicherte Standort kann ausgelesen und mit dem eigentlichen Standort verglichen werden.
Bei Gleichheit wird die Stoppuhr angehalten und die Zeitdifferenz festgehalten.
Das kann durch Tools wie Espresso getestet werden.
Durch ausreichend viele Wiederholungen, zum Beispiel 20, kann somit die benötigte Zeit bis zur Aktualisierung bei Standortänderung statistisch repräsentativ getestet werden.