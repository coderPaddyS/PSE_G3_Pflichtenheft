\section{Funktionssequenzen}

Für alle Funktionssequenzen gilt:
Offlinefunktionalität ist auch bei Online-Nutzung vorhanden und Funktionalität, für die man nicht angemeldet sein muss, ist auch bei Anmeldung vorhanden. Dabei ist der Anmeldeprozess die einzige Ausnahme.
Folgende Funktionssequenzen sind zu überprüfen:

\paragraph{\Gls{Benutzer} offline}
\begin{enumerate}[label=\textbf{/T\arabic*0/}, align=left]
	\item \label{/T10/} Testfall zu \hyperref[/FA40/]{/FA40/}: Suchen von Gebäuden, Räumen durch \Glspl{Standardbezeichner}.
	\item \label{/T20/} Testfall zu \hyperref[/FA40/]{/FA40/}: Suchen von Gebäuden, Räumen durch \gls{lokal} hinzugefügte \Glspl{Alias}.
	\item \label{/T30/} Testfall zu \hyperref[/FA40/]{/FA40/}: Suchen von Gebäuden, Räumen durch bereits heruntergeladene, für alle Benutzer sichtbare \Glspl{Alias}.
	\item \label{/T40/} Testfall zu \hyperref[/FA70/]{/FA70/}: \Glspl{Alias} für Gebäude, Räume \gls{lokal} auf dem Smartphone hinzufügen.
	\item \label{/T50/} Testfall zu \hyperref[/FA52/]{/FA52/}, \hyperref[/FA54/]{/FA54/}, \hyperref[/FA55/]{/FA55/}, \hyperref[/FA57/]{/FA57/}: Auswahl von Kategorien \Glspl{Theme}, Suchverlauf, Impressum, Datenschutz, Hilfe.
	\item[\textbf{/T51/}] \label{/T51/} Testfall zu \hyperref[/FA51/]{/FA51/}: App mit deutscher, englischer, anderer Systemsprache öffnen.
	\item \label{/T60/} Testfall zu \hyperref[/FA53/]{/FA53/}: Löschen des Suchverlaufs.
	\item \label{/T70/} Testfall zu \hyperref[/FA80/]{/FA80/}: Entfernen von \gls{lokal} gespeicherten \Glspl{Alias}n.
	\item \label{/T80/} Testfall zu \hyperref[/FA31/]{/FA31/}: Anfordern des Standortzugriffs.
	\item \label{/T90/} Testfall zu \hyperref[/FA11/]{/FA11/}: Anzeigen der \Gls{Kartenansicht}.
	\item \label{/T100/} Testfall zu \hyperref[/FA31/]{/FA31/}, \hyperref[/FA32/]{/FA32/}: Anzeigen des aktuellen Standorts.
	\item \label{/T110/} Testfall zu \hyperref[/FA31/]{/FA31/}: Anzeigen von Straßennamen und Kreuzungen beim Anzeigen des aktuellen Standorts.
	\item \label{/T120/} Testfall zu \hyperref[/FA21/]{/FA21/}: Anzeige und Interaktion der \Gls{Etagenkarte}.
	\item[\textbf{/T121/}] \label{/T121/} Testfall zu \hyperref[/FA24/]{/FA24/}: Auswählen von Gebäuden auf der \Gls{Karte}, ändern der Etage, auf eine leere Stelle auf der \Gls{Karte} klicken. Erneut auf das Gebäude klicken.
	\item \label{/T130/} Testfall zu \hyperref[/FA14/]{/FA14/}: Auswählen von Gebäuden auf der \Gls{Karte}.
	\item \label{/T140/} Testfall zu \hyperref[/FA23/]{/FA23/}: Auswählen von Räumen auf der \Gls{Etagenkarte}.
	\item \label{/T150/} Testfall zu \hyperref[/FA22/]{/FA22/}: Wechseln der Etagen auf der \Gls{Etagenkarte}.
	\item \label{/T160/} Testfall zu \hyperref[/FA12/]{/FA12/}: Verschieben der \Gls{Karte} mit einem Finger.
	\item \label{/T170/} Testfall zu \hyperref[/FA13/]{/FA13/}: Hineinzoomen in die \Gls{Karte} durch Auseinanderführen von zwei Fingern.
	\item \label{/T180/} Testfall zu \hyperref[/FA13/]{/FA13/}: Herauszoomen aus der \Gls{Karte} durch Zusammenführen von zwei Fingern.
	\item \label{/T190/} Testfall zu \hyperref[/FA14/]{/FA14/}: Anzeigen von \gls{lokal} gespeicherten Informationen zu dem ausgewählten Gebäude bzw. Raum.
\end{enumerate}

\paragraph{\Gls{Benutzer} online, nicht angemeldet}
\begin{enumerate}[label=\textbf{/T\arabic*0/}, align=left, resume]
	\item \label{/T200/} Testfall zu \hyperref[/FA40/]{/FA40/}: Suchen von Gebäuden, Räumen durch für alle \Glspl{Benutzer} sichtbare \Glspl{Alias}.
	\item \label{/T210/} Testfall zu \hyperref[/FA40/]{/FA40/}: Suchen von Räumen über Personennamen.
	\item \label{/T220/} Testfall zu \hyperref[/FA14/]{/FA14/}, \hyperref[/FA23/]{/FA23/}: Anzeigen von allen Informationen zu dem ausgewählten Gebäude bzw. Raum.
	\item \label{/T230/} Testfall zu \hyperref[/FA60/]{/FA60/}: Anmelden von nicht angemeldeten Benutzern, die ein Konto bei einer unterstützenden Anmeldeplattform besitzen. Der Anmeldevorgang muss für alle unterstützenden Anmeldeplattformen der Konten möglich sein.
\end{enumerate}

\paragraph{\Gls{Benutzer} online, angemeldet}
\begin{enumerate}[label=\textbf{/T\arabic*0/}, align=left, resume]
	\item \label{/T240/} Testfall zu \hyperref[/FA70/]{/FA70/}: Vorschläge für \Glspl{Alias} von Gebäuden, Räumen hinzufügen.
	\item \label{/T250/} Testfall zu \hyperref[/FA90/]{/FA90/}: Vorschläge für \Glspl{Alias} von Gebäuden, Räumen bewerten.
\end{enumerate}

\paragraph{\Gls{Administrator} online}
\begin{enumerate}[label=\textbf{/T\arabic*0/}, align=left, resume]
	\item \label{/T260/} Testfall zu \hyperref[/FA104/]{/FA104/}: Standardwert der Mindestanzahl positiver oder negative Bewertungen der \Glspl{Alias-Vorschlag} festlegen.
	\item \label{/T270/} Testfall zu \hyperref[/FA103/]{/FA103/}: Temporär die Mindestanzahl positiver oder negativer Bewertungen von \Glspl{Alias-Vorschlag} auswählen, damit diese dem \Gls{Administrator} angezeigt werden.
	\item \label{/T280/} Testfall zu \hyperref[/FA102/]{/FA102/}, \hyperref[/FA105/]{/FA105/}, \hyperref[/FA106/]{/FA106/}, \hyperref[/FA107/]{/FA107/}: \Glspl{Alias-Vorschlag} einsehen, annehmen, ablehnen, auf die \Gls{Blacklist} setzen.
	\item \label{/T290/} Testfall zu \hyperref[/FA108/]{/FA108/}, \hyperref[/FA109/]{/FA109/}, \hyperref[/FA110/]{/FA110/}: \Gls{Blacklist} einsehen, Begriffe hinzufügen und entfernen.
	\item[\textbf{/T291/}] \label{/T291/} Testfall zu \hyperref[/FA120/]{/FA120/}: \Glspl{Alias} vorschlagen, die Wörter enthalten, die auf der \Gls{Blacklist} stehen. \Gls{Alias} vorschlagen, die keine Wörter der \Gls{Blacklist} enthalten, vorschlagen.
	\item[\textbf{/T292/}] \label{/T292/} Testfall zu \hyperref[/FA130/]{/FA130/}: Einen \Gls{Benutzer} zum \Gls{Administrator} setzen, dieser versucht sich im \Gls{Admin-Panel} anzumelden.
	\item[\textbf{/T293/}] \label{/T293/} Testfall zu \hyperref[/FA140/]{/FA140/}: Einen \Gls{Administrator} entfernen, dieser versucht sich im \Gls{Admin-Panel} anzumelden.
	\item[\textbf{/T294/}] \label{/T294/} Testfall zu \hyperref[/FA111/]{/FA111/}: Einen \gls{global}en \Gls{Alias} entfernen und erneut vorschlagen.
	\item[\textbf{/T295/}] \label{/T295/} Testfall zu \hyperref[/FA112/]{/FA112/}: Eine Änderung bestätigen bzw. revidieren.
\end{enumerate}
