\section{Funktionssequenzen}

Für alle Funktionssequenzen gilt:
Offlinefunktionalität ist auch bei Online-Nutzung vorhanden und Funktionalität, für die man nicht angemeldet sein muss, ist auch bei Anmeldung vorhanden. Dabei ist der Anmeldeprozess die einzige Ausnahme.
Folgende Funktionssequenzen sind zu überprüfen:

\paragraph{\Gls{Benutzer} offline}
\begin{enumerate}[label=\textbf{/T\arabic*0/}, align=left]
	\item Suchen von Gebäuden, Räumen durch \Glspl{Standardbezeichner}.
	\item Suchen von Gebäuden, Räumen durch \gls{lokal} hinzugefügte \Glspl{Alias}.
	\item Suchen von Gebäuden, Räumen durch bereits heruntergeladene, für alle Benutzer sichtbare \Glspl{Alias}.
	\item \Glspl{Alias} für Gebäude, Räume \gls{lokal} auf dem Smartphone hinzufügen.
	\item Auswahl von Kategorien \Glspl{Theme}, Sprachen, Suchverlauf, Impressum, Hilfe.
	\item Löschen des Suchverlaufs.
	\item Entfernen von \gls{lokal} gespeicherten \Glspl{Alias}n.
	\item Anfordern des Standortzugriffs.
	\item Anzeigen der \Gls{Kartenansicht}.
	\item Anzeigen des aktuellen Standorts.
	\item Anzeigen von Straßennamen und Kreuzungen beim Anzeigen des aktuellen Standorts.
	\item Anzeige und Interaktion der \Gls{Etagenkarte}.
	\item Auswählen von Gebäuden auf der \Gls{Karte}.
	\item Auswählen von Räumen auf der \Gls{Etagenkarte}.
	\item Wechseln der Etagen auf der \Gls{Etagenkarte}.
	\item Verschieben der \Gls{Karte} mit einem Finger.
	\item Hineinzoomen in die \Gls{Karte} durch Auseinanderführen von zwei Fingern.
	\item Herauszoomen aus der \Gls{Karte} durch Zusammenführen von zwei Fingern.
	\item Anzeigen von \gls{lokal} gespeicherten Informationen zu dem ausgewählten Gebäude bzw. Raum.
\end{enumerate}

\paragraph{\Gls{Benutzer} online, nicht angemeldet}
\begin{enumerate}[label=\textbf{/T\arabic*0/}, align=left, resume]
	\item Suchen von Gebäuden, Räumen durch für alle \Glspl{Benutzer} sichtbare \Glspl{Alias}.
	\item Suchen von Räumen über Personennamen.
	\item Anzeigen von allen Informationen zu dem ausgewählten Gebäude bzw. Raum.
	\item Anmelden.
\end{enumerate}

\paragraph{\Gls{Benutzer} online, angemeldet}
\begin{enumerate}[label=\textbf{/T\arabic*0/}, align=left, resume]
	\item Vorschläge für \Glspl{Alias} von Gebäuden, Räumen hinzufügen.
	\item Vorschläge für \Glspl{Alias} von Gebäuden, Räumen bewerten.
\end{enumerate}

\paragraph{\Gls{Administrator} online}
\begin{enumerate}[label=\textbf{/T\arabic*0/}, align=left, resume]
	\item Standardwert der Mindestanzahl positiver oder negative Bewertungen der \Glspl{Alias-Vorschlag} festlegen.
	\item Temporär die Mindestanzahl positiver oder negativer Bewertungen von \Glspl{Alias-Vorschlag} auswählen, damit diese dem \Gls{Administrator} angezeigt werden.
	\item \Glspl{Alias-Vorschlag} einsehen, annehmen, ablehnen, auf die \Gls{Blacklist} setzen.
	\item \Gls{Blacklist} einsehen, Begriffe hinzufügen und entfernen.
\end{enumerate}
