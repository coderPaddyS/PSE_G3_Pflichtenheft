\section[Ortungsfunktion]{/FA30/ Ortungsfunktion}
\label{/FA30/}
\paragraph{Ziel}
Der Standort des \Gls{Benutzer} wird auf der \Gls{Karte} angezeigt.
\paragraph{Kategorie}
primär
\paragraph{Zugehörige Kriterien}
\ref{/MK80/}
\paragraph{Verfügbarkeit}
\gls{offline}, alle \Glspl{Benutzer}
\paragraph{Vorbedingung}
Der \Gls{Benutzer} befindet sich in der Hauptansicht (\Gls{Kartenansicht}).
\paragraph{Nachbedingung}
Der \Gls{Benutzer} befindet sich in der Hauptansicht (\Gls{Kartenansicht}).
\paragraph{Akteure}
angemeldeter oder nicht angemeldeter \Gls{Benutzer}

\subsection*{/FA31/ Ortung auf Aufforderung}
\label{/FA31/}
\paragraph{Auslösendes Ereignis}
Der \Gls{Benutzer} klickt auf den Knopf zur Ortung.
\paragraph{Beschreibung}
\begin{enumerate}
    \item Wenn die App keinen Standortzugriff besitzt, wird dieser angefordert.
    \item Wird der Standortzugriff verweigert, passiert nichts. Die normale \Gls{Kartenansicht} bleibt ohne Änderung bestehen.
    \item Erhält die App Standortzugriff oder besitzt sie ihn bereits, wird der aktuelle Standort auf der \Gls{Karte} angezeigt. Die \Gls{Karte} wird so verschoben, dass sich der aktuelle Standort in der Mitte befindet. Es wird soweit in die \Gls{Karte} hineingezoomt, dass Straßennamen und Kreuzungen erkennbar sind.
\end{enumerate}

\subsection*{/FA32/ Ortung beim Start der Applikation}
\label{/FA32/}
\paragraph{Auslösendes Ereignis}
Der \Gls{Benutzer} öffnet die App und gelangt das erste Mal (für diese Session) in die \Gls{Kartenansicht}.
\paragraph{Beschreibung}
\begin{enumerate}
    \item Besitzt die App keinen Standortzugriff, wird die Position auf der \Gls{Karte} angezeigt, die zuletzt angezeigt wurde, bevor die App geschlossen wurde. Liegt keine letzte Position vor, wird der Platz zwischen der Mensa am Adenauerring und der \Gls{KIT}-Bibliothek angezeigt.
    \item Besitzt die App Standortzugriff, wird der aktuelle Standort auf der \Gls{Karte} angezeigt. Der aktuelle Standort befindet sich in der Mitte der angezeigten \Gls{Karte}. Straßennamen und Kreuzungen sind erkennbar.
\end{enumerate}