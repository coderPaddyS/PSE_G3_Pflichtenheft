\section[Entfernen eines Administratorzugangs]{/FA140/ Entfernen eines Administratorzugangs}
\label{/FA140/}
Diese Anforderung wird durch den Testfall \hyperref[/T320/]{/T320/} getestet.
\paragraph{Ziel}
Dem \Gls{Server} wird ein bestehender \Gls{Administrator}zugang entfernt.
Dieser Administrator kann sich nicht mehr im \Gls{Admin-Panel} authentifizieren.
Dieser Administrator kann sich nicht mehr mit dem \Gls{Server} durch \Gls{SSH} verbinden.
\paragraph{Kategorie}
primär
\paragraph{Zugehörige Kriterien}
\hyperref[/MK110/]{/MK110/}, \hyperref[/MK112/]{/MK112/}
\paragraph{Verfügbarkeit}
online
\paragraph{Vorbedingung}
\Gls{Server} ist eingeschaltet und mit dem Internet verbunden.
Ein bestehender \Gls{Administrator}zugang soll entfernt werden.
Ein \Gls{Administrator} hat eine \Gls{Kommandozeile} geöffnet und ist mit dem Internet verbunden.
Diese \Gls{Kommandozeile} kann \Gls{SSH} ausführen.
Dem \Gls{Administrator} liegt der Nutzername des zu entfernenden \Gls{Administrator}s vor.
\paragraph{Nachbedingung}
\Gls{Server} ist eingeschaltet und mit dem Internet verbunden.
Der bestehende \Gls{Administrator}zugang wurde entfernt.
Der bestehende \Gls{Administrator} kann sich nicht mehr im \Gls{Admin-Panel} authentifizieren.
Der bestehende \Gls{Administrator} kann sich nicht mehr durch \Gls{SSH} mit dem Server verbinden.
\paragraph{Akteure}
\Gls{Server}
Neuer \Gls{Administrator}
Alter \Gls{Administrator}
\paragraph{Auslösendes Ereignis}
Ein bestehender \Gls{Administrator} soll entfernt werden.
\paragraph{Beschreibung}
\begin{enumerate}
    \item   Ein \Gls{Administrator} verbindet sich durch \Gls{SSH} mit dem Server.
            Hierfür wird auf der Konsole \texttt{ssh pse@itermori.de} eingegeben.
    \item   Dieser \Gls{Administrator} entfernt den bestehenden \Gls{Administrator}zugang.
            Hierfür wird auf der Konsole \texttt{./delad <Nutzername>} eingegeben.
            \texttt{<Nutzername>} steht hier für den Nutzernamen des bestehenden \Gls{Administrator}s.
    \item   Der bestehende Administrator trennt seine Verbindung mit dem \Gls{Server} durch die Eingabe \texttt{exit}.
\end{enumerate}
