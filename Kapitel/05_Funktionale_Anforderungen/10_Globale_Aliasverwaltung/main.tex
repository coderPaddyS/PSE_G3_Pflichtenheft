\section{Globale Aliasverwaltung}
\label{Aliasverwaltungsfunktion}
\paragraph{Ziel}
\Glspl{Administrator}en können \Glspl{Alias-Vorschlag} und  \Glspl{Alias} verwalten.
\paragraph{Kategorie}
sekundär
\paragraph{Verfügbarkeit}
online, \Glspl{Administrator}
\paragraph{Vorbedingung}
\Gls{Admin-Panel} ist geöffnet.
\paragraph{Nachbedingung}
\Gls{Admin-Panel} ist geöffnet.
\paragraph{Akteure}
\Gls{Administrator}, \Gls{Server}
\paragraph{Auslösendes Ereignis}
keines
\paragraph{Beschreibung}
\begin{enumerate}[start=100, label=\textbf{/FA\arabic*/}, align=left]
    \item Das \Gls{Admin-Panel} ist passwortgeschützt.
    \item Das \Gls{Admin-Panel} zeigt eingereichte \Glspl{Alias-Vorschlag} mit ihrer Bewertung an.
    \item Der \Gls{Administrator} kann auswählen, wie viele positive Bewertungen ein \Gls{Alias-Vorschlag} haben muss, damit er angezeigt wird.
    \item Der \Gls{Administrator} kann den Standardwert wählen, wie viele positive Bewertungen ein \Gls{Alias-Vorschlag} haben muss, damit er angezeigt wird. Der neue Standardwert wird gespeichert.
    \item Der \Gls{Administrator} kann einen \Gls{Alias-Vorschlag} annehmen. Der \Gls{Alias-Vorschlag} ist nun ein \Gls{Alias}.
    \item Der \Gls{Administrator} kann einen \Gls{Alias-Vorschlag} ablehnen. Der \Gls{Alias-Vorschlag} wird verworfen.
    \item Der \Gls{Administrator} kann einen \Gls{Alias-Vorschlag} ablehnen und auf eine \Gls{Blacklist} setzen.
    \item Der \Gls{Administrator} kann einen beliebigen Begriff auf die \Gls{Blacklist} setzen.
    \item Der \Gls{Administrator} kann die \Gls{Blacklist} einsehen.
    \item Der \Gls{Administrator} kann einen Begriff von der \Gls{Blacklist} löschen.
    \item Der \Gls{Administrator} kann einen existierenden \Gls{Alias} entfernen.
\end{enumerate}