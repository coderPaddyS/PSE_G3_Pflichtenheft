\section{Globale Aliasverwaltung}
\label{Aliasverwaltungsfunktion}
\paragraph{Ziel}
Administratoren können Alias-Vorschläge verwalten.
\paragraph{Kategorie}
sekundär
\paragraph{Verfügbarkeit}
online, Administratoren
\paragraph{Vorbedingung}
Admin-Panel ist geöffnet.
\paragraph{Nachbedingung}
Admin-Panel ist geöffnet.
\paragraph{Akteure}
Administrator, Server
\paragraph{Auslösendes Ereignis}
keines
\paragraph{Beschreibung}
\begin{enumerate}[start=100, label=\textbf{/FA\arabic*/}, align=left]
    \item Das Admin-Panel ist passwortgeschützt.
    \item Das Admin-Panel zeigt eingereichte Alias-Vorschläge mit ihrer Bewertung an.
    \item Der Administrator kann auswählen, wie viele positive Bewertungen ein Vorschlag haben muss, damit er angezeigt wird.
    \item Der Administrator kann den Standardwert wählen, wie viele positive Bewertungen ein Vorschlag haben muss, damit er angezeigt wird. Der neue Standardwert wird gespeichert.
    \item Der Administrator kann einen Vorschlag annehmen. Der Vorschlag ist nun ein Alias.
    \item Der Administrator kann einen Vorschlag ablehnen. Der Vorschlag wird verworfen.
    \item Der Administrator kann einen Vorschlag ablehnen und auf eine Blacklist setzen.
    \item Der Administrator kann einen beliebigen Begriff auf die Blacklist setzen.
    \item Der Administrator kann die Blacklist einsehen.
    \item Der Administrator kann einen Begriff von der Blacklist löschen.
\end{enumerate}