\section[Globale Aliasverwaltung]{/FA100/ Globale Aliasverwaltung}
\label{/FA100/}
\paragraph{Ziel}
\Glspl{Administrator} können \Glspl{Alias-Vorschlag} und  \Glspl{Alias} verwalten.
\paragraph{Kategorie}
sekundär
\paragraph{Zugehörige Kriterien}
\hyperref[/MK60/]{/MK60/}, \hyperref[/MK61/]{/MK61/}, \hyperref[/MK62/]{/MK62/}
\paragraph{Verfügbarkeit}
online, \Glspl{Administrator}
\paragraph{Vorbedingung}
\Gls{Admin-Panel} ist geöffnet.
\paragraph{Nachbedingung}
\Gls{Admin-Panel} ist geöffnet.
\paragraph{Akteure}
\Gls{Administrator}, \Gls{Server}
\paragraph{Auslösendes Ereignis}
keines
\paragraph{Beschreibung}
\begin{enumerate}[start=101, label=\textbf{/FA\arabic*/}, align=left]
    \item Das \Gls{Admin-Panel} ist passwortgeschützt.
    \item \label{/FA102/} Das \Gls{Admin-Panel} zeigt eingereichte \Glspl{Alias-Vorschlag} mit ihrer Bewertung an. Diese funktionale Anforderung wird durch den Testfall \hyperref[/T280/]{/T280/} getestet.
    \item \label{/FA103/} Der \Gls{Administrator} kann auswählen, wie viele positive oder negative Bewertungen ein \Gls{Alias-Vorschlag} haben muss, damit er angezeigt wird. Diese funktionale Anforderung wird durch den Testfall \hyperref[/T260/]{/T260/} getestet.
    \item \label{/FA104/} Der \Gls{Administrator} kann den Standardwert wählen, wie viele positive oder negative Bewertungen ein \Gls{Alias-Vorschlag} haben muss, damit er angezeigt wird. Der neue Standardwert wird gespeichert. Diese funktionale Anforderung wird durch den Testfall \hyperref[/T260/]{/T260/} getestet.
    \item \label{/FA105/} Der \Gls{Administrator} kann einen \Gls{Alias-Vorschlag} annehmen. Der \Gls{Alias-Vorschlag} wird vom \Gls{Server} in die \Gls{Alias}-\Gls{Datenbank} aufgenommen. Der \Gls{Alias-Vorschlag} ist nun ein \gls{global}er \Gls{Alias}. Diese funktionale Anforderung wird durch den Testfall \hyperref[/T280/]{/T280/} getestet.
    \item \label{/FA106/} Der \Gls{Administrator} kann einen \Gls{Alias-Vorschlag} ablehnen. Der \Gls{Alias-Vorschlag} wird verworfen. Diese funktionale Anforderung wird durch den Testfall \hyperref[/T280/]{/T280/} getestet.
    \item \label{/FA107/} Der \Gls{Administrator} kann einen \Gls{Alias-Vorschlag} ablehnen und auf eine \Gls{Blacklist} setzen. Der \Gls{Server} fügt den {Alias-Vorschlag} zur \Gls{Blacklist} hinzu. Der \Gls{Alias-Vorschlag} wird verworfen. Diese funktionale Anforderung kann durch den Testfall \hyperref[/T280/]{/T280/} getestet.
    \item \label{/FA108/} Der \Gls{Administrator} kann einen beliebigen Begriff auf die \Gls{Blacklist} setzen. Diese funktionale Anforderung wird durch den Testfall \hyperref[/T290/]{/T290/} getestet.
    \item \label{/FA109/} Der \Gls{Administrator} kann die \Gls{Blacklist} einsehen. Diese funktionale Anforderung wird durch den Testfall \hyperref[/T290/]{/T290/} getestet.
    \item \label{/FA110/} Der \Gls{Administrator} kann einen Begriff von der \Gls{Blacklist} löschen. Diese funktionale Anforderung wird durch den Testfall \hyperref[/T290/]{/T290/} getestet.
    \item \label{/FA111/} Der \Gls{Administrator} kann einen existierenden \gls{global}en \Gls{Alias} entfernen. Dabei löscht der \Gls{Server} den \gls{global}en \Gls{Alias} aus der \Gls{Alias}-\Gls{Datenbank}. Dies wird durch Testfall \hyperref[/T294/]{/T294/} getestet.
    \item \label{/FA112/} Der \Gls{Administrator} muss Änderungen an den \Glspl{Alias-Vorschlag}n, den \gls{global}en \Glspl{Alias}n und der \Gls{Blacklist} bestätigen, bevor diese durchgeführt werden. Er kann diese auch wieder revidieren. Diese funktionale Anforderung wird durch den Testfall \hyperref[/T295/]{/T295/} getestet.
\end{enumerate}