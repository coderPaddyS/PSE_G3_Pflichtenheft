\section{Etagenkarteninteraktion}
\paragraph{Ziel}
Benutzer können mit Etagenkarten interagieren.
\paragraph{Kategorie}
primär
\paragraph{Verfügbarkeit}
offline (ggf. eingeschränkt: Information zu Bürobesitzer liegt ggf. nur online vor), alle Benutzer
\paragraph{Vorbedingung}
Für das ausgewählte Gebäude liegen Etagenkarten vor.
\paragraph{Nachbedingung}

\paragraph{Akteure}
angemeldeter oder nicht angemeldeter Benutzer
\paragraph{Auslösendes Ereignis}
\begin{itemize}
    \item Der Benutzer wählt eine Gebäude im Kartenmodus aus.
    \item Die Suche nach einem Raum war erfolgreich.
\end{itemize}
 
\paragraph{/FA20/ Anzeigen der Etagenkarte}
\label{/FA20/}
\begin{enumerate}
    \item Die App zeigt die Etagenkarte auf dem Gebäude an. Die normale Karte ist weiterhin sichtbar und benutzbar. Die Etagenkarte enthält die Räume auf der Etage. Gegebenenfalls kann eine Etagenkarte auch nur die Hörsäle eines Gebäudes enthalten.
    \item Die App zeigt an, die wievielte Etage gerade angezeigt wird.
\end{enumerate}
\paragraph{Beschreibung der Interaktionen}
\begin{enumerate}[start=21, label=\textbf{/FA\arabic*/}, align=left]
    \item Der Benutzer kann die angezeigte Etage wechseln.
    \item Klickt der Benutzer auf einen Raum, so zeigt die App Informationen zu diesem Raum wie Raumnummer und um wessen Büro es sich handelt (sofern diese Information vorliegt) an. Die Karte ist weiterhin sichtbar. Klickt der Benutzer erneut (ggf. auf eine andere Stelle) auf die Karte schließen sich die Informationen.
    \item Klickt der Benutzer neben das Gebäude auf die Karte schließt sich die Etagenansicht und die normale Kartenansicht wird angezeigt. Kehrt der Benutzer direkt wieder in die Etagenansicht des gleichen Gebäudes zurück, so wird die Etagenkarte der Etage angezeigt, die vor dem Verlassen angezeigt wurde.
\end{enumerate}