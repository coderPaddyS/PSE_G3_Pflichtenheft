\section[Etagenkarteninteraktion]{/FA20/ Etagenkarteninteraktion}
\label{/FA20/}
\paragraph{Ziel}
\Glspl{Benutzer} können mit \Glspl{Etagenkarte} interagieren.
\paragraph{Kategorie}
primär
\paragraph{Zugehörige Kriterien}
\hyperref[/MK70/]{/MK70/} \hyperref[/MK91/]{/MK91/} \hyperref[/MK92/]{/MK92/} \hyperref[/MK93/]{/MK93/} \hyperref[/KK50/]{/KK50/} \hyperref[/KK100/]{/KK100/}
\paragraph{Verfügbarkeit}
\gls{offline}, alle \Glspl{Benutzer}
\paragraph{Vorbedingung}
Für das ausgewählte Gebäude liegen \Glspl{Etagenkarte} vor.
\paragraph{Nachbedingung}
keine
\paragraph{Akteure}
angemeldeter oder nicht angemeldeter \Gls{Benutzer}
\paragraph{Auslösendes Ereignis}
\begin{itemize}
    \item Der \Gls{Benutzer} wählt ein Gebäude in der \Gls{Kartenansicht} aus.
    \item Die Suche nach einem Raum war erfolgreich.
\end{itemize}
 
\paragraph{/FA21/ Anzeigen der Etagenkarte}
\label{/FA21/}
\begin{enumerate}
    \item Die App zeigt die \Gls{Etagenkarte} auf dem Gebäude an. Die normale \Gls{Karte} ist weiterhin sichtbar und benutzbar. 
    Die \Gls{Etagenkarte} zeigt die Räume auf der Etage. An jedem Raum steht die Raumnummer.
    Gegebenenfalls kann eine \Gls{Etagenkarte} auch nur die Hörsäle eines Gebäudes anzeigen.
    \item Die App zeigt an, die wievielte Etage gerade angezeigt wird.
    \item Die \Gls{Etagenkarte} ist \gls{offline} verfügbar.Die \Gls{Etagenkarte} und die Kartendaten (Raumnummern, Personendaten) sind daher \gls{lokal} auf dem \Gls{Endgeraet} gespeichert.
\end{enumerate}
\paragraph{Beschreibung der Interaktionen}
\begin{enumerate}[start=22, label=\textbf{/FA\arabic*/}, align=left]
    \item Der \Gls{Benutzer} kann die angezeigte Etage wechseln.
    \item Klickt der \Gls{Benutzer} auf einen Raum, so zeigt die App Informationen zu diesem Raum wie Raumnummer 
    und um wessen Büro es sich handelt (sofern diese Information vorliegt) an. 
    Die \Gls{Karte} ist weiterhin sichtbar. 
    Klickt der \Gls{Benutzer} erneut (ggf. auf eine andere Stelle) auf die \Gls{Karte} schließen sich die Informationen.
    \item Klickt der \Gls{Benutzer} neben das Gebäude auf die \Gls{Karte} schließt sich die \Gls{Etagenkartenansicht} und die normale \Gls{Kartenansicht} wird angezeigt. Kehrt der \Gls{Benutzer} direkt wieder in die \Gls{Etagenkartenansicht} des gleichen Gebäudes zurück, so wird die \Gls{Etagenkarte} der Etage angezeigt, die vor dem Verlassen angezeigt wurde.
\end{enumerate}