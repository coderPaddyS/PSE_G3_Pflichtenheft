\section[Karteninteraktion]{/FA10/ Karteninteraktion}
\label{/FA10/}
\paragraph{Ziel}
\Glspl{Benutzer} können mit der \Gls{Karte} des \Gls{Campus} interagieren.
\paragraph{Kategorie}
primär
\paragraph{Zugehörige Kriterien}
\hyperref[/MK90/]{/MK90/}, \hyperref[/KK100/]{/KK100/}
\paragraph{Verfügbarkeit}
\gls{offline}, alle \Glspl{Benutzer}
\paragraph{Vorbedingung}
keine
\paragraph{Nachbedingung}
keine
\paragraph{Akteure}
angemeldeter oder nicht angemeldeter \Gls{Benutzer}
\paragraph{Auslösendes Ereignis}
Der \Gls{Benutzer} startet die Applikation oder kehrt aus einem andern Menü zur \Gls{Kartenansicht} zurück.
\paragraph{/FA11/ Anzeigen der Karte}\label{/FA11/}
\begin{enumerate}
    \item Die App zeigt einen Ausschnitt einer \Gls{Karte}. Die verfügbare \Gls{Karte} deckt mindestens einen Radius von 5 km um den \Gls{Campus} ab. Auf der \Gls{Karte} werden Straßen, und Gebäude des \Gls{Campus} angezeigt. Gebäude des \Gls{Campus} sind auf der \Gls{Karte} hervorgehoben. An jedem Gebäude steht die Gebäudenummer.
    \item Die \Gls{Karte} ist \gls{offline} verfügbar. Die \Gls{Karte} und die Kartendaten (Straßennamen, Gebäude, Gebäudenummern, Gebäudeadresse) sind daher \gls{lokal} auf dem \Gls{Endgeraet} gespeichert.
\end{enumerate}
\paragraph{Interaktionen}
\begin{enumerate}[start=12, label=\textbf{/FA\arabic*/}, align=left]
    \item Der \Gls{Benutzer} kann mit einem Finger die \Gls{Karte} verschieben. Die \Gls{Karte} wird synchron mit dem Finger verschoben. Dabei bleibt die Stelle der \Gls{Karte}, auf der sich der Finger befindet, beim Verschieben unter dem Finger.
    \item Der \Gls{Benutzer} kann durch das Zusammenführen von zwei Fingern aus der \Gls{Karte} herauszoomen. Der \Gls{Benutzer} kann durch das Auseinanderführen von zwei Fingern in die \Gls{Karte} hineinzoomen.
    \item Der \Gls{Benutzer} kann auf Gebäude klicken. Die App zeigt Informationen zum Gebäude an. 
    Die Informationen beinhalten mindestens Gebäudenummer und Adresse. Die \Gls{Karte} ist dabei weiterhin sichtbar. 
    Klickt der \Gls{Benutzer} auf eine andere Stelle auf der \Gls{Karte} verschwinden die Informationen. 
    Wenn \Glspl{Etagenkarte} für das Gebäude vorliegen, wechselt die \Gls{Karte} beim (ersten) Klicken auf das Gebäude zusätzlich in die \Gls{Etagenkartenansicht}.
\end{enumerate}