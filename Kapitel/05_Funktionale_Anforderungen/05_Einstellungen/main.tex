\section[Einstellungen]{/FA50/ Einstellungen}
\label{/FA50/}
\paragraph{Ziel}
\Gls{Benutzer} können die Einstellungen der App festlegen.
\paragraph{Kategorie}
sekundär
\paragraph{Zugehörige Kriterien}
\hyperref[/KK90/]{/KK90/}
\paragraph{Verfügbarkeit}
\gls{offline}, alle \Glspl{Benutzer}
\paragraph{Vorbedingung}
Der \Gls{Benutzer} befindet sich in der Hauptansicht (\Gls{Kartenansicht}).
\paragraph{Nachbedingung}
Einstellungen wurden angepasst.
\paragraph{Akteure}
angemeldeter oder nicht angemeldeter \Gls{Benutzer}
\paragraph{Auslösendes Ereignis}
Der \Gls{Benutzer} öffnet die Einstellungen.
\paragraph{Beschreibung}
\begin{enumerate}[start=51, label=\textbf{/FA\arabic*/}, align=left]
    \item \label{/FA52/} Der \Gls{Benutzer} kann ein Anzeige-\Gls{Theme} auswählen. Diese funktionale Anforderung wird durch den Testfall \hyperref[/T50/]{/T50/} getestet.
    \item \label{/FA53/} Der \Gls{Benutzer} kann den \Gls{Suchverlauf} löschen. Diese funktionale Anforderung wird durch den Testfall \hyperref[/T60/]{/T60/} getestet.
    \item \label{/FA54/} Der \Gls{Benutzer} kann die Speicherung des \Gls{Suchverlauf}s aktivieren und deaktivieren. Der Standartwert ist \dq aktiviert\dq{}. Diese funktionale Anforderung wird durch den Testfall \hyperref[/T50/]{/T50/} getestet.
    \item \label{/FA55/} Der \Gls{Benutzer} kann sich das Impressum der App anzeigen lassen. Diese funktionale Anforderung wird durch den Testfall \hyperref[/T50/]{/T50/} getestet.
    \item \label{/FA56/} Der \Gls{Benutzer} kann sich Informationen zum Datenschutz der App anzeigen lassen. Dies wird durch Testfall \hyperref[/T50/]{/T50/} getestet.
    \item \label{/FA57/} Der \Gls{Benutzer} kann sich Erklärungen zu verschiedenen Funktionen der App anzeigen lassen. Diese funktionale Anforderung wird durch den Testfall \hyperref[/T50/]{/T50/} getestet.
\end{enumerate}