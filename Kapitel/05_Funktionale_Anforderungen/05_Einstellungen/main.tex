\section{Einstellungen}
\paragraph{Ziel}
\Gls{Benutzer} können die Einstellungen der App festlegen.
\paragraph{Kategorie}
sekundär
\paragraph{Verfügbarkeit}
offline, alle Benutzer
\paragraph{Vorbedingung}
Der Benutzer befindet sich in der Hauptansicht (Kartenansicht).
\paragraph{Nachbedingung}
Einstellungen wurden angepasst.
\paragraph{Akteure}
angemeldeter oder nicht angemeldeter Benutzer
\paragraph{Auslösendes Ereignis}
Der Benutzer öffnet die Einstellungen.
\paragraph{Beschreibung}
\begin{enumerate}[start=50, label=\textbf{/FA\arabic*/}, align=left]
    \item Der \Gls{Benutzer} kann eine der unterstützten Sprachen auswählen. Alle Texte in der App werden in der ausgewählten Sprache angezeigt.
    \item Der \Gls{Benutzer} kann ein Anzeige-Theme auswählen.
    \item Der \Gls{Benutzer} kann den Suchverlauf löschen.
    \item Der \Gls{Benutzer} kann die Speicherung des Suchverlaufes aktivieren und deaktivieren. Der Standartwert ist "aktiviert".
    \item Der \Gls{Benutzer} kann sich das Impressum der App anzeigen lassen.
    \item Der \Gls{Benutzer} kann sich Informationen zum Datenschutz der App anzeigen lassen.
    \item Der \Gls{Benutzer} kann sich Erklärungen zu verschiedenen Funktionen der App anzeigen lassen.
\end{enumerate}