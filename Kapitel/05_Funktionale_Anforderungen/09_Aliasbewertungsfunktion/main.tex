\section[Alias-Vorschläge bewerten]{/FA90/ Alias-Vorschläge bewerten}
\label{/FA90/}
Diese funktionale Anforderung wird durch die Testfälle \hyperref[/T250/]{/T250/}, \hyperref[/T330/]{/T330/}, \hyperref[/T360/]{/T360/} getestet.
\paragraph{Ziel}
\Glspl{Benutzer} können positives oder negatives Feedback zu \Glspl{Alias-Vorschlag}n geben.
\paragraph{Kategorie}
sekundär
\paragraph{Zugehörige Kriterien}
\hyperref[/KK120/]{/KK120/}
\paragraph{Verfügbarkeit}
online, angemeldete \Glspl{Benutzer}
\paragraph{Vorbedingung}
Ein Gebäude oder Raum ist ausgewählt.
\paragraph{Nachbedingung}
Die App hat das Feedback an den \Gls{Server} gesendet.
\paragraph{Akteure}
angemeldeter \Gls{Benutzer}, \Gls{Server}
\paragraph{Auslösendes Ereignis}
Informationen zu einem Raum oder Gebäude werden angezeigt.
\paragraph{Beschreibung}
\begin{enumerate}
    \item Die App stellt eine Anfrage an den \Gls{Server}.
    \item Der \Gls{Server} sendet aktuelle \Glspl{Alias-Vorschlag} für das Gebäude/den Raum an die App.
    \item Die App zeigt mehrere \Glspl{Alias-Vorschlag} an. Es werden keine \Glspl{Alias-Vorschlag} angezeigt, die der \Gls{Benutzer} bereits bewertet hat. Es werden keine \Glspl{Alias-Vorschlag} angezeigt, die der \Gls{Benutzer} selbst vorgeschlagen hat.
    \item Der \Gls{Benutzer} gibt positives oder negatives Feedback zu einem \Gls{Alias-Vorschlag}.
    \item Die App sendet das Feedback an den \Gls{Server}.
    \item Die App zeigt den bewerteten \Gls{Alias-Vorschlag} nicht weiter an.
\end{enumerate}