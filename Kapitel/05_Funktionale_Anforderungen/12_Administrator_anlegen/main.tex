\section[Anlegen eines Administratorzugangs]{/FA130/ Anlegen eines Administratorzugangs}
\label{/FA130/}
Diese Anforderung wird durch den Testfall \hyperref[/T292/]{/T292/} getestet.
\paragraph{Ziel}
Dem \Gls{Server} wird ein neuer \Gls{Administrator}zugang hinzugefügt.
Dieser Administrator kann sich nun im \Gls{Admin-Panel} authentifizieren.
Dieser Administrator kann sich nun mit dem \Gls{Server} durch \Gls{SSH} verbinden.
\paragraph{Kategorie}
primär
\paragraph{Zugehörige Kriterien}
\hyperref[/MK110/]{/MK110/}, \hyperref[/MK111/]{/MK111/}
\paragraph{Verfügbarkeit}
online
\paragraph{Vorbedingung}
Der neue \Gls{Administrator} hat sich in der App angemeldet.
\Gls{Server} ist eingeschaltet und mit dem Internet verbunden.
Die Mehrheit unter den \Glspl{Administrator} möchten einen neuen Administrator hinzufügen.
Ein bestehender \Gls{Administrator} hat eine \Gls{Kommandozeile} geöffnet und ist mit dem Internet verbunden.
Diese \Gls{Kommandozeile} kann \Gls{SSH} ausführen.
Dem bestehenden \Gls{Administrator} liegt der Nutzername und einen \Gls{Public-Key} des neuen Administrators vor.

\paragraph{Nachbedingung}
\Gls{Server} ist eingeschaltet und mit dem Internet verbunden.
Der neue \Gls{Administrator} wurde hinzugefügt.
Der neue \Gls{Administrator} kann sich im \Gls{Admin-Panel} authentifizieren.
Der neue \Gls{Administrator} kann sich durch \Gls{SSH} mit dem Server verbinden.

\paragraph{Akteure}
\Gls{Server}
Neuer \Gls{Administrator}, Bestehender \Gls{Administrator}
\paragraph{Auslösendes Ereignis}
Ein neuer \Gls{Administrator} soll hinzugefügt werden.
\paragraph{Beschreibung}
\begin{enumerate}
    \item   Der bestehende \Gls{Administrator} verbindet sich durch \Gls{SSH} mit dem Server.
            Hierfür wird auf der Konsole \texttt{ssh pse@itermori.de} eingegeben.
    \item   Der bestehende \Gls{Administrator} legt den neuen Administrator-Zugang an.
            Hierfür wird auf der Konsole \texttt{./regad <Nutzername> <Public-Key>} eingegeben.
            \texttt{<Nutzername>} steht hier für den Nutzernamen des neuen \Gls{Administrator}s
            und \texttt{<Public-Key>} steht für den \Gls{Public-Key} des neuen \Gls{Administrator}s.
    \item   Der bestehende Administrator trennt seine Verbindung mit dem \Gls{Server} durch die Eingabe \texttt{exit}.
\end{enumerate}
