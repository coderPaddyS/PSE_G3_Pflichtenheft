\section[Aliasüberprüfung]{/FA120/ Aliasüberprüfung}
\label{/FA120/}
Diese Anforderung wird durch Testfall \hyperref[/T291/]{/T291/} abgedeckt.
\paragraph{Ziel}
Der \Gls{Server} sortiert \Glspl{Alias-Vorschlag} aus, die verboten oder schon vorhanden sind.
\paragraph{Kategorie}
sekundär
\paragraph{Zugehörige Kriterien}

\paragraph{Verfügbarkeit}
online
\paragraph{Vorbedingung}
\Gls{Server} ist eingeschaltet und mit dem Internet verbunden.
\paragraph{Nachbedingung}
\Gls{Server} ist eingeschaltet. Der \Gls{Alias-Vorschlag} wurde aussortiert oder zur Bewertung freigegeben.
\paragraph{Akteure}
\Gls{Server}
\paragraph{Auslösendes Ereignis}
Der \Gls{Server} hat einen \Gls{Alias-Vorschlag} erhalten.
\paragraph{Beschreibung}
Der \Gls{Server} überprüft eingehende \Glspl{Alias-Vorschlag}. Der \Gls{Server} sortiert \Glspl{Alias-Vorschlag} aus, auf die mindestens einer der folgenden Punkte zutrifft:
\begin{itemize}
    \item Der \Gls{Alias-Vorschlag} ist bereits ein existierender \Gls{Alias}.
    \item Der \Gls{Alias-Vorschlag} existiert bereits (als \Gls{Alias-Vorschlag}).
    \item Der \Gls{Alias-Vorschlag} steht auf der \Gls{Blacklist}.
\end{itemize}
Aussortierte \Glspl{Alias-Vorschlag} werden nicht zur Bewertung durch angemeldete \Glspl{Benutzer} freigegeben. Der \Gls{Server} löscht aussortierte \Glspl{Alias-Vorschlag} nach spätestens einer Woche. Nicht aussortierte \Glspl{Alias-Vorschlag} werden zur Bewertung durch angemeldete \Glspl{Benutzer} freigegeben.
