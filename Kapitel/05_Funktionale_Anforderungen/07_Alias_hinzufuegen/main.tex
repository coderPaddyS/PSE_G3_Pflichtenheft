\section[Alias hinzufügen]{/FA70/ Alias hinzufügen}
\label{/FA70/}
Diese funktionale Anforderung wird durch die Testfälle \ref{/T40/}, \ref{/T240/}, \ref{/T320/}, \ref{/T340/}, \ref{/T350/} getestet.
\paragraph{Ziel}
Der \Gls{Benutzer} kann einen \Gls{Alias} \gls{lokal} hinzufügen (nur für den \Gls{Benutzer} selbst sichtbar). Der \Gls{Benutzer} kann einen \Gls{Alias} vorschlagen (für alle \Gls{Benutzer} sichtbar).
\paragraph{Kategorie}
sekundär
\paragraph{Zugehörige Kriterien}
\hyperref[/MK50/]{/MK50/}
\paragraph{Verfügbarkeit}
\begin{itemize}
    \item \Gls{lokal}es hinzufügen: \gls{offline}, alle \Glspl{Benutzer}
    \item Vorschlagen: online, nur angemeldete \Glspl{Benutzer}
\end{itemize}

\paragraph{Vorbedingung}
Gebäude oder Raum ist ausgewählt.
\paragraph{Nachbedingung}
\Gls{Alias} ist \gls{lokal} gespeichert und wenn gewünscht zusätzlich als \Gls{Alias-Vorschlag} eingereicht.
\paragraph{Akteure}
angemeldeter oder nicht angemeldeter \Gls{Benutzer}
\paragraph{Auslösendes Ereignis}
\Gls{Benutzer} wählt das Hinzufügen eines \Gls{Alias} aus.
\paragraph{Beschreibung}
\begin{enumerate}
    \item Die Tastatur öffnet sich.
    \item Der \Gls{Benutzer} gibt den gewünschten \Gls{Alias} ein.
    \item Der \Gls{Benutzer} wählt \gls{lokal}es Hinzufügen oder zusätzliches Vorschlagen als \gls{global}en \Gls{Alias} für alle \Glspl{Benutzer} aus.
    \item Die App prüft, ob der \Gls{Alias} bereits existiert.
    \subitem{Alias existiert:} Die App zeigt eine Fehlermeldung an.
    \subitem{Alias existiert nicht:} \Gls{Alias} wird \gls{lokal} gespeichert und, wenn ausgewählt, als \Gls{Alias-Vorschlag} an den \Gls{Server} gesendet.
\end{enumerate}
