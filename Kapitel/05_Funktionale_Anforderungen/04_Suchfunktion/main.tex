\section[Suchfunktion]{/FA40/ Suchfunktion}
\label{/FA40/}
\paragraph{Ziel}
Der \Gls{Benutzer} kann nach Gebäuden und Räumen suchen. 
Das Suchen ist möglich mit:
\begin{itemize}
    \item Gebäudenummer
    \item Gebäudenummer und Raumnummer
    \item Name einer Person (wenn nach Büro gesucht wird)
    \item \gls{lokal}er oder \gls{global}er \Gls{Alias}
\end{itemize}

\paragraph{Kategorie}
primär
\paragraph{Verfügbarkeit}
\gls{offline} (eingeschränkt: Suche mit Personenname ist ggf. nur online möglich; Abgleich mit neuen \Glspl{Alias}n, die noch nicht heruntergeladen wurden, ist \gls{offline} nicht möglich), alle \Glspl{Benutzer}
\paragraph{Vorbedingung}
Der \Gls{Benutzer} befindet sich in der Hauptansicht (\Gls{Kartenansicht}).
\paragraph{Nachbedingung (Erfolg)}
Das gesuchte Objekt wird angezeigt.
\paragraph{Nachbedingung (Misserfolg)}
Es erscheint eine Fehlermeldung. Existiert das Gebäude aber der gesuchte Raum nicht, wird zusätzlich das Gebäude angezeigt.
\paragraph{Akteure}
angemeldeter oder nicht angemeldeter \Gls{Benutzer}
\paragraph{Auslösendes Ereignis}
Der \Gls{Benutzer} klickt auf das \Gls{Suchfeld}.
\paragraph{Beschreibung}
\begin{enumerate}
    \item Die Tastatur öffnet sich. Die App schlägt die letzten \Glspl{Suchbegriff} vor.
    \item Der \Gls{Benutzer} tippt einen \Gls{Suchbegriff} ein.
    \item Die App zeigt \Glsplural{Alias-Vorschlag} zum eingegebenen Text an. Die \Glspl{Alias-Vorschlag} werden nach jedem eingegebenen Zeichen aktualisiert.
    \item Der \Gls{Benutzer} wählt einen \Gls{Alias-Vorschlag} aus oder bestätigt die Eingabe.
    \item Die App prüft, ob zur Eingabe ein Treffer existiert.
        \begin{enumerate}
            \item{Erfolg:} Die App zeigt das gesuchte Gebäude bez. den gesuchten Raum an.
            \item{Misserfolg:} 
            \begin{enumerate}
                \item Wenn kein Gebäude zum eingegebenen \Gls{Suchbegriff} existiert, oder die eingegebene Person nicht im System existiert, wird eine Fehlermeldung angezeigt.
                \item Existiert das Gebäude, aber der Raum nicht, zeigt die App das Gebäude und eine Fehlermeldung an.
            \end{enumerate}
        \end{enumerate}
\end{enumerate}
