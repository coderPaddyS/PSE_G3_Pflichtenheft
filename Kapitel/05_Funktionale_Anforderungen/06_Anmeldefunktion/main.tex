\section{/FA60/ Anmeldefunktion}
\label{Anmeldefunktion}
\label{/FA60/}
\paragraph{Ziel}
Nicht angemeldete \Gls{Benutzer} können sich anmelden.
\paragraph{Kategorie}
sekundär
\paragraph{Verfügbarkeit}
online, nicht angemeldete Benutzer
\paragraph{Vorbedingung}
Der \Gls{Benutzer} besitzt ein Konto bei einer unterstützten Anmeldeplattform (vorzugsweise Shibboleth Identity Provider).
\paragraph{Nachbedingung (Erfolg)}
Der \Gls{Benutzer} besitzt eine eindeutige \Gls{ID}. Der \Gls{Benutzer} ist ein "`angemeldeter \Gls{Benutzer}"'.
\paragraph{Nachbedingung (Misserfolg)}
Die App zeigt eine Fehlermeldung an.
\paragraph{Akteure}
nicht angemeldeter \Gls{Benutzer}, Anmeldeplattform
\paragraph{Auslösendes Ereignis}
\begin{itemize}
      \item Der \Gls{Benutzer} hat auf eine "Anmelden"-Schaltfläche geklickt.
      \item Der \Gls{Benutzer} hat versucht eine Aktion auszuführen, die nur angemeldeten \Glspl{Benutzer}n zur Verfügung steht.
\end{itemize}
\paragraph{Beschreibung}
\begin{enumerate}
      \item Der \Gls{Benutzer} kann den Anmeldevorgang in jedem Schritt abbrechen. Dann kehrt die App zur vorherigen Ansicht zurück.
      \item Die App weist auf den Datenschutz hin.
      \item Der \Gls{Benutzer} akzeptiert den Datenschutz.
      \item Der \Gls{Benutzer} wählt eins der unterstützten Anmeldeverfahren.
      \item Der \Gls{Benutzer} meldet sich über das gewählte Verfahren an.
            \begin{itemize}
                  \item{Erfolg:} Die App speichert eine eindeutige \Gls{ID} für den \Gls{Benutzer}. Die App weist dem \Gls{Benutzer} die Rolle "angemeldeter \Gls{Benutzer}" zu.
            \end{itemize}
      \item Die App zeigt Informationen zum Erfolg oder Misserfolg der Anmeldung an.
      \item Die App kehrt zur vorherigen Ansicht zurück. 
\end{enumerate}