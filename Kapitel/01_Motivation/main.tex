\chapter{Motivation}

\paragraph{}
Vor allem für Studierende im ersten Semester oder Gäste am Karlsruher Institut für Technologie (KIT) 
kann es schwierig sein, sich ohne Unterstützung auf dem Universitätsgelände zurechtzufinden. Bisherige 
technologische Hilfsmittel wie der "Campusplan"\footnote{\href{https://www.kit.edu/campusplan/}{https://www.kit.edu/campusplan/}} oder sonstige Applikationen
zur Kartennutzung unterstützen weder das Finden von Zielpunkten über häufig genutzte \Gls{Alias} noch kann der Standort der meisten Räume oder Büros ermittelt werden.
Unsere Applikation soll daher sowohl das allgemeine Hinzufügen verbreiteter Alias, als auch private und somit lokale
Benennungen unterstützen. Ebenso soll die Orientierung innerhalb von Gebäuden durch Etagenangeben und gegebenenfalls auch
Etagenkarten erleichtert werden. Hierbei sollen Räume idealerweise nicht nur über ihre Nummer gefunden werden können.
Häufig gesuchte Räumlichkeiten wie Büros oder universitäre Behörden sollen sowohl über offizielle Namen als auch
über hinzugefügte Alias gefunden werden können.
Insgesamt ist es unser Ziel, jedem Nutzer ohne besondere Kenntnisse, welcher unsere Anwendung auf seinem 
Android-Gerät verwendet, sein Zielort auf dem Campus zu finden, selbst wenn ihm dieser nur in "Studenten-Jargon"
genannt wurde. 