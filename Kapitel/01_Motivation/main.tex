\chapter{Motivation}

Vor allem für Studierende im ersten Semester oder Gäste am Karlsruher Institut für Technologie (KIT) kann es schwierig sein, 
sich ohne Unterstützung auf dem Universitätsgelände zurechtzufinden. 
Bisherige technologische Hilfsmittel wie der "Campusplan"\footnote{\href{https://www.kit.edu/campusplan/}{https://www.kit.edu/campusplan/}} oder sonstige Applikationen 
zur Kartennutzung unterstützen weder das Finden von Zielpunkten über häufig genutzte \Glspl{Alias} noch kann der Standort der meisten Räume oder Büros ermittelt werden.\\
Unsere Applikation soll daher sowohl das allgemeine Hinzufügen verbreiteter \Glspl{Alias}, als auch private und somit \gls{lokal}e Benennungen unterstützen.
Ebenso soll die Orientierung innerhalb von Gebäuden durch Etagenangaben und gegebenenfalls auch 
\Glspl{Etagenkarte} erleichtert werden. Hierbei sollen Räume idealerweise nicht nur über ihre Nummer gefunden werden können.
Häufig gesuchte Räumlichkeiten wie Büros oder universitäre Behörden sollen sowohl über offizielle Namen als auch 
über hinzugefügte \Glspl{Alias} gefunden werden können.\\
Insgesamt ist es unser Ziel, jedem Nutzer, welcher unsere Anwendung auf seinem \Gls{Android}-Gerät verwendet, es zu ermöglichen, 
ohne besondere Vorkenntnisse, seinen Zielort auf dem \Gls{Campus} finden zu können, selbst wenn ihm dieser nur in \dq Studenten-Jargon\dq{} genannt wurde. 