\subsection{Suchen des Gebäudes 50.34}

Der \Gls{Benutzer} Max ist sich unsicher, wo er sich gerade genau auf dem \Gls{Campus} befindet.
Zudem muss er dringend zu dem Gebäude 50.34 gehen.
Er weiß aber auch nicht, wo das Gebäude liegt.
Max öffnet die App, und ist nun in der Hauptansicht (\Gls{Kartenansicht}) der App.
Er sieht, dass ihm die App einen Knopf zur \Gls{Ortung} zeigt.
Max drückt auf diesen Knopf.
Die App besitzt jedoch keinen Standortzugriff.
Dementsprechend fordert die App zuerst den Standortzugriff an.
Max ist einverstanden, und gewährt somit der App den Standortzugriff.
Die App zeigt daraufhin den Standort von Max an.
Er befindet sich in der Nähe der \Gls{KIT}-Bibliothek.
Nun klickt Max auf das ebenfalls angezeigten \Gls{Suchfeld}, und die Tastatur seines Smartphones öffnet sich.
Er fängt daraufhin an, in das \Gls{Suchfeld} die \Gls{Zeichenkette} \texttt{50.34} einzutippen. 
Während des Eintippens erscheinen Max Vorschläge, um seine Eingabe zu vervollständigen. 
Er schreibt jedoch die \Gls{Zeichenkette} ohne Hilfe zu Ende. 
Nun drückt er auf die Eingabetaste. 
Die App prüft die Eingabe \texttt{50.34} von Max, und nimmt sie an. 
So wird auf der Karte nun das Gebäude 50.34, \Gls{KIT} Fakultät für Informatik, angezeigt. 
Zudem wird die entsprechende Adresse des Gebäudes angezeigt, \textit{Am Fasanengarten 5}. 
Max macht sich nun auf dem Weg zu Gebäude 50.34.