\subsection{Bewerten vom \Gls{Alias-Vorschlag} Infobau für das Gebäude 50.34}

Der Benutzer Max öffnet die App.
Er hat außerdem aktuell Internetzugang.
Max sucht auf der \Gls{Kartenansicht} nach dem Gebäude 50.34.
Die App findet es, und stellt zugleich eine Anfrage an den \Gls{Server}, indem sie die Informationen zu Gebäude 50.34 abruft.
Nachdem die Anfrage vom \Gls{Server} abgearbeitet ist, werden die Informationen zu Gebäude 50.34 einschließlich aktuelle, noch offene \Glspl{Alias-Vorschlag} zu dem Gebäude angezeigt.
\Glspl{Alias-Vorschlag}, die Max selbst unterbreitet hat, werden nicht angezeigt.
Max kann nun den aktuellen, noch offenen \Gls{Alias-Vorschlag} \texttt{Infobau} bewerten.
Er hat die Option, den \Gls{Alias-Vorschlag} positiv oder negativ zu bewerten.
Max findet den \Gls{Alias} \texttt{Infobau} für Gebäude 50.34 sinnvoll.
Dementsprechend möchte er den \Gls{Alias-Vorschlag} positiv bewerten.
Doch er hat vergessen sich anzumelden.
Seine Bewertung schlägt fehl, da er sich zuerst anmelden muss.
Max hat sich noch nie in der App angemeldet.
Demzufolge bekommt er einen Hinweis über das verwendete Anmeldeverfahren.
Auch erhält er einen Hinweis über die betroffenen und gesammelten Daten.
Die App fragt Max, ob er mit der Anmeldung fortfahren möchte.
Max ist einverstanden, fährt fort mit der Anmeldung und das Anmeldeverfahren wird somit eingeleitet.
Er meldet sich erfolgreich an und wird zur \Gls{Kartenansicht} des Gebäudes 50.34 samt den \Glspl{Alias-Vorschlag} zurückgeleitet.
Max bewertet nun den gewünschten \Gls{Alias-Vorschlag} \texttt{Infobau} positiv.
Seine Bewertung schlägt diesmal nicht fehl.
Die App sendet die positive Rückmeldung von Max an den \Gls{Server}.
Der \Gls{Server} aktualisiert die Bewertung des \Gls{Alias-Vorschlag}s \texttt{Infobau} für das Gebäude 50.34.
Der \Gls{Alias-Vorschlag} hat nun 10 positive und keine negative Bewertungen.
Die App zeigt Max nun den bewerteten \Gls{Alias-Vorschlag} nicht mehr an.
Der Administrator Arnold meldet sich auf dem \Gls{Admin-Panel} des \Gls{Server} mit seinem Passwort an.
Die App zeigt Arnold den eingereichten \Gls{Alias-Vorschlag} \texttt{Infobau} an.
Denn er hatte zuvor festgelegt, dass ab 10 positive Bewertungen ein \Gls{Alias-Vorschlag} auf dem \Gls{Admin-Panel} angezeigt wird.
Arnold ist ebenfalls mit dem \Gls{Alias-Vorschlag} einverstanden.
Er entscheidet sich dementsprechend, den \Gls{Alias-Vorschlag} anzunehmen.
Nun ist der \Gls{Alias-Vorschlag} \texttt{Infobau} in der App tatsächlich auch ein \Gls{Alias} vom Gebäude 50.34.
Tippt demzufolge ab sofort die Benutzerin Henriette \texttt{Infobau} in der Suchfunktion ein, findet die App das Gebäude 50.34 und zeigt es an.
