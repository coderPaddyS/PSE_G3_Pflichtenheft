\subsection{Interagieren mit der Etagenkarte vom Gebäude 50.34}

Der \Gls{Benutzer} Max hat zwei Termine vereinbart.
Einerseits hat er einen Termin mit Prof. Dr. Ralf H. Reussner vereinbart.
Andererseits hat er einen Termin mit M.Sc. Erik Burger vereinbart.
Er weiß, dass er zu den Büroräumen der beiden \Gls{KIT}-Angestellten gehen muss.
Auch weiß Max, dass die Büroräume im selben Gebäude sind.
Allerdings weiß er nicht, um welche Büroräume es sich genau handelt und in welchem Gebäude sie sind.
Seine Freundin Erika, die schonmal zu den Büroraum von M.Sc. Erik Burger gehen musste, erinnert sich leider nur daran, dass der Büroraum von M.Sc. Erik Burger im 2. OG vom Gebäude 50.34 ist.
Er öffnet die App und befindet sich auf der Hauptansicht (\Gls{Kartenansicht}).
Er verwendet die Suchfunktion, um zunächst nach dem Büroraum von Prof. Dr. Ralf H. Reussner zu suchen.
Max gibt demzufolge die \Gls{Zeichenkette} \texttt{Prof. Dr. Ralf H. Reussner} ein und bestätigt seine Eingabe.
Die App findet den Büroraum von Prof. Dr. Ralf H. Reussner, nämlich Raum 327 im Gebäude 50.34.
Sie zeigt dementsprechend die \Gls{Etagenkartenansicht} von Gebäude 50.34 und den Büroraum 327.
Des Weiteren zeigt die App die Etage, in der sie sich gerade befindet, nämlich das 3. Obergeschoss (OG).
Jetzt, dass Max den Büroraum von Prof. Dr. Ralf H. Reussner kennt, sucht er nach dem Büroraum von M.Sc. Erik Burger.
In der \Gls{Etagenkartenansicht} von Gebäude 50.34 wechselt er zu der Etage 2. OG.
Nachdem Max auf das Info-\Gls{Icon} von einigen Räumen gedrückt hat, klickt er auf das Info-\Gls{Icon} vom Raum 239.
Die App zeigt an, dass es sich um den Büroraum von M.Sc. Erik Burger handelt.
Nun weiß Max, um welche Büroräume es sich genau handelt.
In der \Gls{Etagenkartenansicht} sieht er noch, wo die Treppe liegt, die in das 2. OG (und 3. OG) führt.
Max macht sich nun auf den Weg zu dem Gebäude 50.34.
