\subsection{Vorschlagen von Info-Bau als Alias für Gebäude 50.34 auf dem Server}

Benutzerin Erika hat auf der App Gebäude 50.34 ausgewählt. Sie hat zudem aktuell auch Internetzugang. Sie klickt auf das von der App angezeigte Aliashinzufügefeld. Die Tastatur öffnet sich. Eine Bestätigungstaste für ihren Vorschlag wird von der App angezeigt, die aber zunächst ausgegraut ist. Außerdem erscheint auch die Option, ob sie ihren eingegebenen Alias lokal speichern oder zusätzlich als Bezeichner für das Gebäude 50.34 für alle Benutzer vorschlagen möchte. Erika tippt auf dem Aliashinzufügefeld die Zeichenkette \texttt{Info-Bau}. Beim Tippen des ersten Zeichens wird ferner die Bestätigungstaste färblich hervorgehoben. Nach dem vollständigen Abtippen der Zeichenkette, wählt sie die Option, ihren eingegebenen Alias zusätzlich als Bezeichner für das Gebäude 50.34 für alle Benutzer zu vorschlagen. Dadurch wird auch die Tastatur geschlossen. Sie drückt nun auf die Bestätigungstaste. Nun prüft die App, ob die Zeichenkette \texttt{Info-Bau} bereits ein Bezeichner für ein Gebäude oder einen Raum ist. Dies ist nicht der Fall, ergo ihr Vorschlag, die Zeichenkette \texttt{Info-Bau} als Bezeichner für das Gebäude 50.34 festzulegen, wird durch die App auf den Server geladen. Zusätzlich speichert die App \texttt{Info-Bau} als Bezeichner für das Gebäude 50.34 lokal auf dem Smartphone von Erika.