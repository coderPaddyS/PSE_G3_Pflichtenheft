\subsection{Suchen des Raums -107 im Gebäude 50.34}

Die Benutzerin Erika studiert Informatik B.Sc. und ist in ihrem ersten Fachsemester.
Es ist Dienstag, sie hat heute zum ersten Mal ihr Tutorium vom Modul Grundbegriffe der Informatik (GBI).
Ihr Tutorium findet im Raum -107 vom Gebäude 50.34 statt.
Sie kennt das Gebäude 50.34, hat aber selbst noch nie das Gebäude betreten.
Erika öffnet die App und befindet sich auf der Hauptansicht (%TODO:\Gls{Kartenansicht}END
).
Als sie auf das Suchfeld klickt, merkt sie, dass die App ihr viele Vorschläge anzeigt.
Denn sie hat zuvor bereits nach einigen %TODO:\Gls{KIT}END
-Gebäuden gesucht.
Erika entscheidet sich demzufolge zuerst in den Einstellungen zu gehen.
Sie öffnet die Einstellungen, die App zeigt ihr unter anderem die Einstellung, ob sie ihren Suchverlauf löschen möchte.
Sie wählt diese Einstellung, womit ihr Suchverlauf gelöscht wird.
Auch sieht sie die Einstellung zur Aktivierung und Deaktivierung der Speicherung vom Suchverlauf.
Erika deaktiviert dies, nun ist sie mit ihren Einstellungen glücklich.
Sie wechselt wieder zur %TODO:\Gls{Kartenansicht}END
und verwendet die Suchfunktion, um nach dem Raum -107 im Gebäude 50.34 zu suchen.
Die Suchfunktion findet den Raum -107 im Gebäude 50.34.
Die %TODO:\Gls{Etagenkarte}END
liegt nämlich für das Gebäude 50.34 vor.
Dementsprechend zeigt die App automatisch die Ansicht der %TODO:\Gls{Etagenkarte}END
und den Raum -107.
Außerdem zeigt sie an, dass gerade die Etage 1. Untergeschoss (UG) zu sehen ist.
Erika weiß nun, dass sie in das 1. UG gehen muss.
Des Weiteren sieht sie in der %TODO:\Gls{Etagenkartenansicht}END
, wo die Treppe ist, die zu das 1. UG führt.
Mit diesem Wissen macht sich Erika nun auf den Weg zu ihrem GBI-Tutorium.