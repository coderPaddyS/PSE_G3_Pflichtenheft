\subsection{Suchen des Raums -107 im Gebäude 50.34}

Die Benutzerin Erika studiert Informatik B.Sc. und ist in ihrem ersten Fachsemester.
Es ist Dienstag, sie hat heute zum ersten Mal ihr Tutorium vom Modul Grundbegriffe der Informatik (GBI).
Ihr Tutorium findet im Raum -107 vom Gebäude 50.34 statt.
Sie kennt das Gebäude 50.34, hat aber selbst noch nie das Gebäude betreten.
Erika öffnet die App und befindet sich auf der Hauptansicht (Kartenansicht).
Sie verwendet die Suchfunktion, um nach dem Raum -107 im Gebäude 50.34 zu suchen.
Die Suchfunktion findet den Raum -107 im Gebäude 50.34.
Die Etagenkarte liegt nämlich für das Gebäude 50.34 vor.
Dementsprechend zeigt die App automatisch die Ansicht der Etagenkarte und den Raum -107.
Außerdem zeigt sie an, dass gerade die Etage 1. Untergeschoss (UG) angezeigt wird.
Erika weiß nun, dass sie in das 1. UG gehen muss.
Des Weiteren sieht sie in der Ansicht der Etagenkarte, wo die Treppe ist, die zu das 1. UG führt.
Mit diesem Wissen macht sich Erika nun auf den Weg zu ihrem GBI-Tutorium.