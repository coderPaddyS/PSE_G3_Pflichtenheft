\subsection{Suchen des Gebäudes 50.34}

Der Benutzer Max ist sich unsicher, wo er sich gerade genau auf dem KIT-Campus befindet.
Zudem muss er dringend zu dem Gebäude 50.34 gehen.
Er weiß aber auch nicht, wo das Gebäude liegt.
Max öffnet die App, und ist nun in der Hauptansicht (Kartenansicht) der App.
Er sieht, dass ihm die App einen Knopf zur Ortung zeigt.
Max drückt auf diesen Knopf.
Die App besitzt sie jedoch keinen Standortzugriff.
Dementsprechend fordert die App zuerst den Standortzugriff an.
Max ist einverstanden, und gewährt somit der App den Standortzugriff.
Die App zeigt daraufhin den Standort von Max an.
Er befindet sich in der Nähe der KIT-Bibliothek.
Nun klickt Max auf der ebenfalls angezeigten Suchfunktion, und die Tastatur seines Smartphones öffnet sich.
Er fängt daraufhin an, auf der Suchfunktion die Zeichenkette \texttt{50.34} abzutippen. 
Während dem Abtippen erscheinen Max Vorschläge, um seine Eingabe zu vervollständigen. 
Er schreibt jedoch die Zeichenkette ohne Hilfe zu Ende. 
Nun drückt er auf die Eingabetaste. 
Die App prüft die Eingabe \texttt{50.34} von Max, und nimmt sie an. 
So wird auf der Karte nun das Gebäude 50.34, KIT Fakultät für Informatik, angezeigt. 
Zudem wird die entsprechende Adresse des Gebäudes angezeigt, \glqq Am Fasanengarten 5\grqq. 
Max macht sich nun auf dem Weg zu Gebäude 50.34.