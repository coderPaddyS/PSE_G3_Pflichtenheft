\subsection{Vorschlagen von Infobau als Alias für Gebäude 50.34 auf dem Server}

Die Benutzerin Erika ist der Meinung, dass viele Studierende das Gebäude 50.34 eher mit dem %TODO:\Gls{Alias}END
\textit{Infobau} assoziieren.
Schließlich sprechen ihre Freunde meistens immer vom \textit{Infobau}, wenn sie das Gebäude 50.34 meinen.
Aktuell findet man das Gebäude 50.34 jedoch nicht mit dem oben erwähnten %TODO:\Gls{Alias}END
.
Erika würde dementsprechend gerne den %TODO:\Gls{Alias}END
vorschlagen.
Auf diese Weise könnte man zukünftig auch nach \texttt{Infobau} suchen, um das Gebäude 50.34 zu finden.
Sie öffnet die App und meldet sich an.
Zudem hat sie aktuell Internetzugang.
Sie befindet sich auf der Hauptansicht (%TODO:\Gls{Kartenansicht}END
) der App, und sucht nach dem Gebäude 50.34.
Erika klickt auf das von der App angezeigte %TODO:\Gls{Alias}END
hinzufügefeld. 
Die Tastatur öffnet sich.
Erika tippt auf dem %TODO:\Gls{Alias}END
hinzufügefeld die %TODO:\Gls{Zeichenkette}END
\texttt{Infobau} ein.
Des Weiteren zeigt die App die Option an, ob sie ihren eingegebenen %TODO:\Gls{Alias}END
lokal speichern oder zusätzlich als Bezeichner für das Gebäude 50.34 für alle Benutzer vorschlagen möchte.
Sie wählt die Option, ihren eingegebenen %TODO:\Gls{Alias}END
zusätzlich als Bezeichner für das Gebäude 50.34 für alle Benutzer vorzuschlagen. 
Dadurch wird auch die Tastatur geschlossen. 
Sie bestätigt nun ihre Eingabe. 
Nun prüft die App, ob die %TODO:\Gls{Zeichenkette}END
\texttt{Infobau} bereits ein Bezeichner für ein Gebäude oder einen Raum ist. 
Dies ist nicht der Fall, ergo ihr Vorschlag, die %TODO:\Gls{Zeichenkette}END
\texttt{Infobau} als Bezeichner für das Gebäude 50.34 festzulegen, wird durch die App auf den %TODO:\Gls{Server}END
geladen. 
Zusätzlich speichert die App \texttt{Infobau} als Bezeichner für das Gebäude 50.34 lokal auf dem Smartphone von Erika.